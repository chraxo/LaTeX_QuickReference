%-------------------- --------------------
%---------- header.tex -------------------
%-------------------- --------------------

\documentclass[
	ngerman,
	a5paper,
	twoside,
	fontsize=9pt,
	notitlepage,
	DIV=9,
	%DIV=calc
	%headings=small,
	%twocolumn,
	%openany,
	version=last,  % neueste Version - Für Rückwärtskompatibilität hier die gewünschte Version eintragen.
]{scrartcl} % scrartcl

%\usepackage[l2tabu, orthodox]{nag} % auf schlechten Code prüfen

%-------------------- --------------------
%------- Pakete laden  -------------------
%-------------------- --------------------
\usepackage{etex}	% damit mit TIKZ kein Fehler "`No room for a new \dimen"' auftaucht.
%%\usepackage 								{fixltx2e} 	% not required with latex version later than 2015
\usepackage [T1] 						{fontenc} 
\usepackage 								{lmodern}
\usepackage [utf8] 					{inputenc}	% (utf8, ansinew, latin1, applemac)
\usepackage [ngerman]				{babel}	
\usepackage 								{ragged2e}	
%\usepackage [babel]					{microtype}

\usepackage{xifthen}
	\newboolean{a5} 	
	%\setboolean{a5}{false}
	\setboolean{a5}{true}

% Layout etc.
\usepackage [singlespacing]	{setspace}
\usepackage [left=1.6cm, right=1cm, top=0.5cm, bottom=00cm, includeheadfoot]	{geometry}
\usepackage									{multicol}
%\usepackage [above, below]	{placeins}
\usepackage[babel]{microtype}

% Formatierungen u.ä.
\usepackage [dvipsnames]		{xcolor}
\usepackage [normalem]			{ulem}
\usepackage									{fancybox}
\usepackage [hyphens]				{url}
\usepackage									{textcomp}

% Mathematik
\usepackage{nicefrac}
\usepackage{amsmath}			
\usepackage{amssymb} 		
\usepackage{mathtools}

% Grafiken
\usepackage{graphicx}	
\usepackage{wrapfig}
\usepackage[hypcap=false]{caption} % hypcap=false, damit die Warnung verschwindet
\usepackage{subcaption}
\usepackage{tikz}
%\usepackage{pgf}

% Tabellen
\usepackage{booktabs}
\usepackage{tabularx}
	%--- Tabularx konfigurieren  -----------
	%Tabellenausrichtung mit Breite-Angabe ermöglichen
	\newcolumntype{L}[1]{>{\RaggedRight \arraybackslash}p{#1}} % linksbündig	%mit tabularx neue Spaltentypen definieren.
	\newcolumntype{M}[1]{>{\RaggedRight \arraybackslash}m{#1}} % linksbündig
	
	\newcolumntype{C}[1]{>{\Centering \arraybackslash}p{#1}} % zentriert
	\newcolumntype{D}[1]{>{\Centering \arraybackslash}m{#1}} % zentriert
		
	\newcolumntype{R}[1]{>{\RaggedLeft \arraybackslash}p{#1}} % rechtsbündig
	
	\newcolumntype{B}[1]{>{\arraybackslash}p{#1}}	%Blocksatz
	%% %% %% %% %% %% %% 
%--------------------------
\usepackage{colortbl}
\usepackage{longtable}
\usepackage{hhline}

% ANDERES
\usepackage{blindtext} 
%\usepackage{pgfplots}
%\usepackage{filecontents}
%\usepackage{enumitem}
\usepackage{xspace}
\usepackage{calc}
\usepackage{scrlayer-scrpage}
%%\usepackage{titlesec}
\usepackage{boxedminipage}


%-------------------- --------------------
%-------------- Listings -----------------
%-------------------- --------------------
\usepackage{listings}
\lstset{literate=%	% Für utf8 notwendig.
	{Ö}{{\"O}}1
	{Ä}{{\"A}}1
	{Ü}{{\"U}}1
	{ß}{{\ss}}1
	{ü}{{\"u}}1
	{ä}{{\"a}}1
	{ö}{{\"o}}1
	{–}{{-}}1			% falls versehentlich – hereinkopiert wird.
	%{~}{{\textasciitilde }}1
}
%\lstset{texcl=true}

%%%
% Farbig markieren:
% - Befehle			(Standard) - erledigt	(blau)
% - Pakete			(teal - grün)
% - Umgebungen	(purple)
% - Längenmaße	
%%%
\lstset{
		keywordstyle=\color{NavyBlue},		% Befehle	% NavyBlue
		emphstyle=[1]{\bfseries\color{purple}},		% Pakete
		emphstyle=[2]{\bfseries\color{teal}},	% Umgebungen
		emphstyle=[3]{\bfseries\color{teal}},	% Längenmaße
}
\lstset{
	numbers			=none,                        % (left, right, none)
	numberstyle	=\tiny,                      	% Line-numbers fonts
	stepnumber	=1,                           % Step between two line-numbers
	numbersep		=5pt,%5pt,       							% How far are line-numbers 
	frame				=none,                        % A frame around the code
	tabsize			=2,                           % Default tab size
	captionpos	=b,                           % Caption-position = bottom
	breaklines	=true,                        % Automatic line breaking?
	breakatwhitespace=false,                	% Automatic breaks only at whitespace?
	showspaces	=false,                       % Dont make spaces visible
	showstringspaces = false									% Leerzeichen in Strings anzeigen
	showtabs		=false,                       % Dont make tabls visible
	columns			=flexible, %flexible,						% Column format  (flexible, fixed or fullflexible)
	%keepspaces=true, 
}

%-----------------------------------------
%% Styles: basicstyle, identifierstyle, commentsyle, stringstyle, keywordstyle
\lstset{
	language		=[LaTeX]{TeX},
	%backgroundcolor=\color{lightgray},
	basicstyle			={\ttfamily\small},
	commentstyle		={\ttfamily\color{darkgray}},
	identifierstyle	={\ttfamily},
	stringstyle			={\ttfamily},
}

%-----------------------------------------
\lstset{
	%keywordstyle=\ttfamily,
	%keywordstyle=\color{blue},		% Keywords font ('*' = uppercase)
	otherkeywords={}, 
	deletekeywords={}
}
%-------- morekeywords -------------------------

%----- morekeywords: Basic LaTeX --------------
\lstset{morekeywords={
part, chapter, subsection, subsubsection, addsec,
ftrq, flq, href,
tableofcontents, thechapter, thesection, thetable, thefigure, chaptername,
newblock, 
chaptermarkformat, sectionmarkformat, subsectionmarkformat, thesubsection, chapappifchapterprefix,
autodot,
setlength,
glqq, grqq, frqq, flqq, glq, grq, frq, grq,		% Anführungszeichen
eqref, appendix, marginline,
phantomsection,	
textsubscript,
texorpdfstring,
}}

%----- morekeywords: SONSTIGES --------------
\lstset{morekeywords={
meinBefehl, beispiel, addlinespace, arraybackslash, xspace, nicefrac, themyctr, mylen, VARIABLE, NAME, text,
zB, addtokomafont,
whiledo,
KOMAoptions,
url,
showkeyslabelformat,
linnumbers,
sectfont,
extrarowheight
}}

%----- morekeywords: Mathe (Package amsmath etc.) ------
\lstset{morekeywords={
mathcal, mathbb, mathring, tag, notag
}}

%----- morekeywords: Package Blindtext --------------
\lstset{morekeywords={
blindtext, Blindtext,
}}

%----- morekeywords: Package booktabs --------------
\lstset{morekeywords={
toprule, midrule, bottomrule, cmidrule
}}

%----- morekeywords: Package calc --------------
\lstset{morekeywords={
real, ratio, maxof, minof
}}

%----- morekeywords: Package Caption --------------
\lstset{morekeywords={
subref, subcaptionref, captionof
}}

%----- morekeywords: Package colortbl --------------
\lstset{morekeywords={
arrayrulecolor, columncolor, rowcolor, cellcolor
}}

%----- morekeywords: Package fancybox --------------
\lstset{morekeywords={
shadowbox, doublebox, ovalbox, Ovalbox
}}

%----- morekeywords: Package graphicx --------------
\lstset{morekeywords={
includegraphics
}}

%----- morekeywords: Package hhline --------------
\lstset{morekeywords={
\hhline
}}

%----- morekeywords: Package listings --------------
\lstset{morekeywords={
lstset, lstinline, lstinputlisting
}}

%----- morekeywords: Package longtable --------------
\lstset{morekeywords={
endfirsthead, endhead, endfoot, endlastfoot, 
}}

%----- morekeywords: Package multicols --------------
\lstset{morekeywords={
columnbreak
}}

%----- morekeywords: Package placeins --------------
\lstset{morekeywords={
FloatBarrier
}}

%----- morekeywords: Package ragged2e --------------
\lstset{morekeywords={
RaggedRight, Centering, RaggedLeft
}}

%----- morekeywords: Package scrpage --------------
\lstset{morekeywords={
clearscrheadfoot, clearscrheadings, clearscrplain,
ihead, chead, ohead, ifoot, cfoot, ofoot,
lehead, cehead, rehead, lohead, cohead, rohead, lefoot, cefoot, refoot, lofoot, cofoot, rofoot,
pagemark, automark, headmark, manualmark,
headfont, footfont, pnumfont,
setheadsepline, setfootsepline, setheadtopline, setfootbotline
}}

%----- morekeywords: Package setspace --------------
\lstset{morekeywords={
doublespacing, onehalfspacing, singlespacing, setstretch
}}

%----- morekeywords: Package tabularx --------------
\lstset{morekeywords={
newcolumntype
}}

%----- morekeywords: Package textcomp --------------
\lstset{morekeywords={
texttwelveudash, textthreequartersemdash,
textdblhyphen, textdblhyphenchar, textasteriskcentered, textborn, textpm, texttimes, textdiv, textfractionsolidus, textonequarter, textonehalf, textthreequarters, textdiscount, textperthousand, textpertenthousand, textsurd,
textonesuperior, texttwosuperior, textthreesuperior, textleftarrow, textrightarrow, textuparrow, textdownarrow, textlangle, textrangle, textlbrackdbl, textrbrackdbl, textlquill, textrquill, textbardbl,
textservicemark, textcopyleft, textcircledP, textpilcrow, textcurrency, textreferencemark, textmusicalnote, textleaf,
textmarried, textdivorced, textdied, textordmasculine, textordfeminine, textbrokenbar, textminus, textasciimacron, textlnot, textinterrobang, textinterrobangdown, texttildelow, textasciibreve, textasciicaron, textacutedbl, textgravedbl, textquotesingle, textquotestraightbase, textquotestraightdblbase, textasciigrave, textasciiacute, textasciidieresis, textasciimacron, textlnot, textinterrobang, textinterrobangdown,  texttildelow,  textasciibreve,  textasciicaron,  textacutedbl,  textgravedbl,  textquotesingle,  textquotestraightbase, textquotestraightdblbase,  textasciigrave, textasciiacute, textasciidieresis,
textblank, textmho, textohm, textmu, textbaht, textcelsius, textcolonmonetary textcentoldstyle, textcent, textdong, textestimated, texteuro, textflorin, textguarani, pounds, textlira, textnaira, textnumero, textpeso, textrecipe, textdollaroldstyle, textyen, textwon,
textopenbullet, textdegree, textbigcircle, textcolonmonetary, textcentoldstyle,
}}

%----- morekeywords: Package tikz --------------
\lstset{morekeywords={
usetikzlibrary, tikz, draw, tikzoptions, node, shade, 
}}

%----- morekeywords: Package ulem --------------
\lstset{morekeywords={
uline, uuline, uwave, sout, xout, dashuline, dotuline,
}}

%----- morekeywords: Package xcolor --------------
\lstset{morekeywords={
textcolor, colorbox, fcolorbox, color, pagecolor, definecolor
}}

%----- morekeywords: Package xifthen --------------
\lstset{morekeywords={
newboolean, setboolean, ifthenelse, isodd, lengthtest, isundefined, equal, AND, OR, NOT, newtest, boolean
}}

%-----------------------------------------


%----- morekeywords: Package VORLAGE --------------
\lstset{morekeywords={
%
}}

%-----------------------------------------
\lstset{
	%% Escape to LaTeX:
	mathescape = false, 	% $a=x$ Mathemodus ermoeglichen
	escapechar = {²},		% Buchstabe zum Verlassen und Zurückkehren (LaTeX-Modus)
	% escapeinside={²}{³},	% Alternative zu escapechar
	%escapebegin= {},		% wird zu Beginn des Escape-Modus eingefügt
	%escapeend= {}				% wird zum Ende des Escape-Modus eingefügt
}

%-----------------------------------------
%----- Weitere Betonungen: Pakete -----
\lstset{emph=[1]{
	usepackage
	%amsmath, amssymb, amsthm, array, babel, blindtext, booktabs, calc, caption, colortbl,  enumitem,  fancybox,  fixltx2e,  mathtools,   microtype,  multicol,   natbib,  nicefrac,  pdflscape,  pdfpages,  pgfplots,  picins,  placeins,  ragged2e, scrpage2  , setspace,  subcaption,  siunitx,  stfloats,  tabularx,  textcomb,  tikz,  url,  ulem,  units,  wrapfig,  xcolor,  xspace, 
}}
%----- Weitere Betonungen: Umbegungen -----
\lstset{
emph=[2]{
	begin, end,
	%array, document, FlushLeft, FlushRight, Center, raggedright, raggedleft, centering, samepage, minipage, figure, wrapfigure, itemize, enumerate, description, tabbing, tabular, lstlisting, math, equation, multline, gather, align, flalign, multicols, landscape, thebibliography,longtable,
},
%emphstyle=[1]{\bfseries\color{purple}}
}

%----- Weitere Betonungen: Längenmaße -----
\lstset{emph=[3]{
	mm, cm, in, pt, em, ex,
}}

%-----------------------------------------
%------------- Listings Ende --------------

\usepackage[pdftex,
	hidelinks,
	pdfauthor=	{cax}, % Autorname
	pdfkeywords={},	% Schluesselwoerter
	pdfdisplaydoctitle=true, 
	pdftitle=		{},	% Titel
	pdfsubject=	{}, % Thema
	%colorlinks=false,
	%linkcolor=MidnightBlue,
	%urlcolor=black,
	%citecolor=black,
	%%%
	bookmarksopen=false,
	%pdffitwindow=true,
	pdfpagelayout={SinglePage},	% (SinglePage, OneColumn, TwoColumnLeft, TwoColumnRight, TwoPageLeft, TwoPageRight)
	pdfpagemode={UseNone},	% (UseNone, UseThumbs, UseOutlines, FullScreen, UseOC, UseAttachments)
	%pagebackref 	=true, 	%zurück-Button im Acrobat-Reader
]{hyperref}  % hyperref sollte als letztes Paket eingebunden werden.%Infos unter {http://de.wikibooks.org/wiki/LaTeX-W%C3%B6rterbuch:_hyperref}


%-------------------- --------------------
%-------------- Einstellungen ------------
%-------------------- --------------------
%% ----- scrheadings (ehemals scrpage2 -----
	\pagestyle{scrheadings}

	\newcommand{\standardKopfzeile}{
		\clearpairofpagestyles
		%\setheadsepline{.4pt}
		\ohead[\pagemark]{\pagemark}
		\rehead[\leftmark]{\leftmark}	% linke Seite innen
		\lohead[\rightmark]{\rightmark} % rechte Seite innen
		\automark[subsection]{section}
	}
	\newcommand{\seitenzahlKopfzeile}{
		\clearpairofpagestyles
		\ohead[\pagemark]{\pagemark}
	}
	\newcommand{\leereKopfzeile}{
		\clearpairofpagestyles
	}
	\newcommand{\AnhangKopfzeile}{
		\clearpairofpagestyles
		\ohead[\pagemark]{\pagemark}
	}
	%\standardKopfzeile
	%\seitenzahlKopfzeile
	\leereKopfzeile
%--------------------------

% ----- Kapitelüberschriften formatieren ------
% --> erst nach dem Inhaltsverzeichnis




%\usepackage{titlesec}
%\titleformat{\section}
%{\color{red}\normalfont\Large\bfseries}
%{\color{Blue}\thesection}{1em}{}
%%(siehe auch http://texblog.org/tag/titlesec/)




%\setlength{\headheight}{0pt}%Überschriften direkt unter Kopfzeile
\pagenumbering{Roman} % Römische Seitennummerierung für Vorspann
	
%%% %%% Trennungs-Liste %%% %%%
	\hyphenation{Pa-ket-be-schrei-bung-en Ka-pi-tel}
%%% %%% %%% %%%

\renewcommand*{\sectfont}{\sffamily\bfseries\boldmath} % Mathe in Überschriften

%\raggedbottom % damit der Inhalt einer nicht füllbaren Seite nicht vertikal verteilt wird. (\raggedbottom, \flushbottom)
\raggedcolumns % (\raggedcolumns, \flushcolumns) damit der Inhalt einer nicht füllbaren Spalte (mittels multicol-Package) nicht vertikal verteilt wird.

\setlength{\extrarowheight}{2pt} % Zeilenabstand in Tabllen

%EOF