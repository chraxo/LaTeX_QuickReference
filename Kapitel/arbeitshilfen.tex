\section{Arbeitshilfen}

\addtolength{\leftmargini}{-10pt}
\begin{itemize}

	\item Blindtext zum Testen von Textausgaben: \lstinline|\usepackage{blindtext}| \Pfeil Befehle \lstinline|\blindtext| (kurzer Text) und \lstinline|\Blindtext| (langer Text).

	\item Zeilennummern zum Korrigieren: \lstinline|\usepackage{lineno} \linnumbers| 

\item Referenz-Keys von Abbildungen, Literaturverzeichniseinträgen etc. am Rand anzeigen: 
	\begin{lstlisting}
\usepackage[notref, notcite]{showkeys}
\usepackage[hyphens]{url}
\usepackage[preserveurlmacro]{breakurl}	% wirklich nötig ???
\renewcommand*\showkeyslabelformat[1]{%	
\fbox{%
	\parbox[t]{\marginparwidth}{
	\raggedright\normalfont\small
	\url{#1}
}}}
	\end{lstlisting}
	
\item Schnelleres kompilieren mittels includeonly: Nach dem Includeonly-Befehl werden nur noch includes berücksichtigt, die in includeonly explizit genannt sind.
\begin{lstlisting}
\includeonly{Kapitel1, Kapitel2}
\end{lstlisting}

\end{itemize}

\hrule \vspace{0.5\baselineskip}
%----- EOF -----