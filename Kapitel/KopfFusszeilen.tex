%%%%%%%%%%%%%%%%%%%%%%%%%%%%%%%%%
\columnbreak
\subsectionstage[3]{Kopf-/ Fußzeilen (\Paket{scrpage2})}
\negAbstand

\begin{lstlisting}
\usepackage{scrlayer-scrpage}
\pagestyle{scrheadings}	% Auf den neuen Seitenstil umschalten
		% Auf der ersten Seite eines Kapitels wird i.d.R. automatisch auf den Seitenstil scrplain umgeschaltet. Auf der nächsten Seite gilt dann wieder scrheadings.
\end{lstlisting}

\noindent\begin{minipage}{\linewidth}

\subsubsectionstage[4]{Löschen der Stile (um "`frisch"' anzufangen)}
\negAbstand

\begin{lstlisting}
 \clearscrheadfoot	% Beide Stile (scrheadings, scrplain) komplett leeren.
 \clearscrheadings	% Stil scrheadings komplett leeren
 \clearscrplain			% Stil scrplain komplett leeren
\end{lstlisting}
\end{minipage}
\negAbstand

\subsubsectionstage[3]{Definition}
\negAbstand

\begin{lstlisting}
\ihead[Text für scrplain-Stil] 
		{Text für scrheadings-Stil}
\ihead{Kopfzeile Innen}
\chead{Kopfzeile Mitte}
\ohead{Kopfzeile Außen}
\ifoot{Fußzeile Innen}
\cfoot{Fußzeile Mitte}
\ofoot{Fußzeile Außen}
\end{lstlisting}

%\negAbstand
%\begin{minipage}{0.7\linewidth}
%\centering
%\fbox{\includegraphics[width=\linewidth]{Bilder/scrguide_S216_scrpage2Befehle}}
%\end{minipage}

\posAbstand
\subsubsectionstage[3]{Befehle für automatischen Text\hspace{-0.5ex}}\vspace{-0.6\baselineskip}
\begin{lstlisting}
\pagemark		% Seitenzahl.
\thepage		% Seitenzahl. (Alternative).
\automark[<rechteSeite>]{<linkeSeite>}	% Definiert \rightmark und \leftmark. (chapter, section, subsection, ...)
\rightmark	% z.B. Kapitel-Name incl.Nr.
\leftmark		% z.B. Section-Name incl.Nr. (\leftmark nur bei Dokumentoption "`twoside"' verfügbar)
\headmark		% 
\manualmark	% 
\end{lstlisting}

\subsubsectionstage[3]{Kapitel-Nummerierung ausschalten}
\label{cha:KapNumAus}
\todo{Quelle}
\negAbstand
\begin{lstlisting}
\renewcommand*{\chaptermarkformat}{} 
\renewcommand*{\sectionmarkformat}{}
\renewcommand*{\subsectionmarkformat}{}

% Orignalwerte  (Nummerierung eingeschaltet): 
\renewcommand{\chaptermarkformat} {\chapappifchapterprefix{\ }\thechapter\autodot\enskip}
\renewcommand*{\sectionmarkformat} {\thesection\autodot\enskip}
\renewcommand*{\subsectionmarkformat} {\thesubsection\autodot\enskip}
\end{lstlisting}

\subsubsectionstage[3]{Schriftformatierung}
\negAbstand
\begin{lstlisting}
\renewcommand*{\headfont} {\normalfont \bfseries}	% Schriftart für Kopf- und Fußzeile
\renewcommand*{\footfont} {\normalfont \itshape} % Von der Kopfzeile abweichende Schriftart für die Fußzeile.
\renewcommand*{\pnumfont} {\normalfont} % Von headfont und footfont abweichende Formatierung für Seitenzahlen
\end{lstlisting}

\subsubsectionstage[3]{Linien}
\setlength{\columnseprule}{0.5pt}
\negAbstand
\begin{lstlisting}
\setheadsepline[\textwidth]{.4pt} % Linie unter der Kopfzeile
\setfootsepline{.4pt} 	% Linie über der Fußnote

\setheadtopline{2pt}[\color{red}]	% Linie über der Kopfzeile
\setfootbotline{2pt}		% Linie unter der Fußnote
\end{lstlisting}

\bigskip
\myhrule
\bigskip