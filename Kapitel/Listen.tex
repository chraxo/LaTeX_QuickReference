%%%%%%%%%%%%%%%%%%
\pagebreak
\sectionstage[2]{Listen / Aufzählungen}

%%%%% itemize %%%%%
\begin{multicols}{2}
\begin{itemize} 
	\item TEXT
	\item TEXT
\end{itemize}

\noindent\begin{minipage}{\linewidth}
\begin{lstlisting}
\begin{itemize} 
	\item TEXT
	\item TEXT
\end{itemize}
\end{lstlisting}
\end{minipage}
\end{multicols}

\hrule

%%%%% enumerate %%%%%
\begin{multicols}{2}
\begin{enumerate} 
	\item TEXT
	\item TEXT
\end{enumerate}

\noindent\begin{minipage}{\linewidth}
\begin{lstlisting}
\begin{enumerate} 
	\item TEXT
	\item TEXT
\end{enumerate}
\end{lstlisting}
\end{minipage}
\end{multicols}

\hrule

%%%%% description %%%%%
\begin{multicols}{2}

\begin{description} 
	\item[First] TEXT
	\item[Second] TEXT
\end{description}

\noindent\begin{minipage}{\linewidth}
\begin{lstlisting}
\begin{description} 
	\item[First] TEXT
	\item[Second] TEXT
\end{description}
\end{lstlisting}
\end{minipage}
\end{multicols}

\vspace{-0.5\baselineskip}
\noindent Einzug verändern:\newline
\lstinline|\addtolength{\leftmargini}{-20pt}|

\subsection{Paket \Paket{enumerate}}
Die Nummerierungsart in der Umgebung \Umgebung{enumerate} kann verändert werden.
%%%\negAbstand
\begin{lstlisting}
\usepackage{enumerate}
\begin{enumerate}[(a)]	% Beispiel
	\item ... % (a), (b), (c)
\end{enumerate}
\end{lstlisting}

\subsection{Paket \Paket{enumitem}}
\negAbstand
Alternative  zu \Paket {enumerate}, jedoch mehr Optionen.
Ermöglicht neue Listenumgebungen, die \emph{einfacher anzupassen} sind.



\hrule \vspace{0.5\baselineskip}
%----- EOF -----
