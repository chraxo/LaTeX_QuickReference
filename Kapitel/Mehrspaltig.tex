%%%%%%%%%%%%%%%%%%%
\setlength{\columnseprule}{0.5pt}

\columnbreak
\subsectionstage[3]{Mehrspaltiger Textsatz (\Paket{multicol})}
\negAbstand

\begin{lstlisting} 
\begin{multicols}{N}		% multicols*:ohne automatischen Spaltenausgleich; N: Anzahl Spalten
	\columnbreak					% manueller Spaltenumbruch
	\hrule								% Horizontale Linie
\end{multicols}
\end{lstlisting}

\subsubsection*{Einstellungen}
\negAbstand

\begin{lstlisting} 
\setlength{\columnseprule}{0.5pt}	% Trennlinie: Strichstärke (optional)
\setlength{\columnsep}{10pt}			% Abstand der Spalten
\end{lstlisting}

\begin{description}

\item[Hinweis] \textcolor{red}{Floats (Gleitumbegungen) sind innerhalb von multicols nicht erlaubt.} 

\item[Hinweis] Werden zwei multicols-Umgebungen ineinander verschachtelt, so wird die Innere Umgebung durch Spaltenumbruch auf die äußere verteilt.

\item[Hinweis] Verschachtelte multicols-Umgebungen können Probleme beim Umbruch bereiten. In diesem Fall sollten die inneren multicols-Umgebungen in mehrere kleinere multicols-Umgebungen aufgesplittet werden.
\end{description}
