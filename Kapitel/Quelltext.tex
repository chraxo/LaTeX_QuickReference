%\columnbreak
\mycolumnbreak


\sectionstage[2]{Quelltexte}
\subsectionstage[4]{verbatim-Umgebung}
\negAbstand
\begin{lstlisting}
	\verb|Direkt ausgegebener Text|
	\begin{verbatim}
		Direkt ausgegebener Text
	\end{verbatim}
\end{lstlisting}
\Befehl{verb} und \Umgebung{verbatim} können nicht als Paramter an andere Befehle übergeben werden.

\subsectionstage[2]{Quelltext (Paket \Paket{listings})}

siehe auch: \url{http://en.wikibooks.org/wiki/LaTeX/Source_Code_Listings}

Für escape-Inside-Option hier nachschlagen: \url{http://tex.stackexchange.com/questions/110268/texcl-escapeinside-and-single-character-comments-with-listings-package}

\subsubsection*{Umlaute bei utf8-Codierung}
{\footnotesize
Bei Verwendung der Kodierung utf8 (\lstinline|\usepackage[utf8]{inputenc}|) können Umlaute (ä ö ü ß) nicht ohne weiteres  dargestellt werden. Abhilfe (Quelle \url{http://uweziegenhagen.de/?p=1500}; \qquad\qquad\qquad 2014):
}
%
\begin{lstlisting}
\lstset{literate=%
    {Ö}{{\"O}}1
    {Ä}{{\"A}}1
    {Ü}{{\"U}}1
    {ß}{{\ss}}1
    {ü}{{\"u}}1
    {ä}{{\"a}}1
    {ö}{{\"o}}1
    %{~}{{\textasciitilde}}1
}
 %ODER:
\lstset{texcl=true} % Schriftsatz mit LaTeX. (Unter Umständen unerwünscht.)
\end{lstlisting}

%\columnbreak
\subsubsection{\Paket{listings}: Einstellungen}
\negAbstand
\lstset{mathescape = true}

\begin{lstlisting}
\usepackage{listings}
\lstset{
		language		=	[LaTeX]{TeX},		%[dialect]{language}
		% Mögliche Sprachen (Auswahl): Assembler, Basic, C++, Delphi, Fortran, Gnuplot, HTML, Java, Mathematica, Matlab, Octave, Perl, PHP, Python, Scilab, SQL, TeX, [LaTeX]{TeX}, VHDL, XML
		%escapeinside = {((**} {**xx},		% ???
		% escapebegin = {((**} 						% ???
		% escapeend		= {**))}						% ???
\end{lstlisting}
\vspace{-1\baselineskip}
\begin{lstlisting}
		basicstyle	=	\ttfamily,
		basicstyle	=	\small,
		identifierstyle=\ttfamily,
		commentstyle=	\ttfamily,
		commentstyle=	\color{gray},
		stringstyle =	\ttfamily,
		keywordstyle=	\ttfamily,
		keywordstyle=	\color{OliveGreen},
\end{lstlisting}
\vspace{-1\baselineskip}
\begin{lstlisting}
		numbers			=	left,  			% (none, left, right)
		numberstyle	=	\tiny, 			
		stepnumber	=	1,    			% Step between two line-numbers
		numbersep		=	5pt,  			% How far are line-numbers from code
		backgroundcolor=\color{white},		
		frame				=	none,       % A frame around the code
		tabsize			=	2,					% Default tab size
		captionpos	=	b,          % Caption-position = bottom
\end{lstlisting}

\vspace{-1\baselineskip}
\begin{lstlisting}
		breaklines	=	true,       % Automatic line breaking?
		breakatwhitespace=false,  % Automatic breaks only at whitespace?
		showspaces	=	false,      % Dont make spaces visible
		showstringspaces = false	% Leerzeichen in Strings anzeigen
		showtabs		=	false,      % Dont make tabls visible
		columns			=	fixed, 			% Column format  (flexible, fixed or fullflexible)
		% morekeywords   = {Wort3, Wort4}
		% deletekeywords = {Wort1, Wort2}

\end{lstlisting}
\vspace{-1\baselineskip}
\begin{lstlisting}
	%	% Escape to LaTeX:
	%	mathescape = true, 	% $a=x$ Mathemodus ermöglichen
		escapechar = {$^2$},		% Buchstabe zum Verlassen und Zurückkehren (LaTeX-Modus)
	% escapeinside={$^2$}{$^2$},	% Alternative zu escapechar
	%	escapebegin= {},		% wird zu Beginn des Escape-Modus eingefügt
	%	escapeend= {}				% wird zum Ende des Escape-Modus eingefügt
	}
\end{lstlisting}
\lstset{mathescape = false}

\noindent\begin{minipage}{\linewidth}
\subsubsection{\Paket{listings}: Verwendung}
\negAbstand

\begin{lstlisting}
\begin{lstlisting}		% Abgesetzter Quellcode
	QUELLCODE
\end {lstlisting}
\end{lstlisting}

\end{minipage}

\begin{lstlisting}
\lstinline|QUELLCODE|	% Inline-Quellcode ohne Zeilenumbrüche
\end{lstlisting}

\begin{lstlisting}
\lstinputlisting[language=Python, firstline=37, lastline=45]{source_filename.py} // Ganze Datei einlesen.
\end{lstlisting}

%\notizenplatz

\hrule \vspace{0.5\baselineskip}
%----- EOF -----
