%%%%%%%%%%%%%%%%%%%
\columnbreak
\sectionstage[2]{Tabellen}
%
\noindent\begin{minipage}{\linewidth}
	%\captionof{table}{TEXT}
	\Centering
	\begin{tabular}{|l|r c l|}
		\hline
			1 &rechts	& mittig& links\\
			2 &a 			& b 		& c \\
		\cline{2-4}
			3 &\multicolumn{3}{c|}{mehrere Spalten}\\
		\hline
	\end{tabular}
\end{minipage}

\begin{lstlisting}
\begin{table}
	\caption{TABELLENÜBERSCHRIFT}
	\label{KEY}
	\centering
	\begin{tabular}{|l|r c l|}
		\hline
			1 &rechts& mittig& links\\
			2 &a & b & c \\
		\cline{2-4}
			3 &\multicolumn{3}{c|}{mehrere Spalten}\\
		\hline
	\end{tabular}
\end{table}
\end{lstlisting}
%
%%%%%%%%%%%%%%%%%%%%%
\subsectionstage[3]{Schönere Tab. (\Paket{booktabs})}

\noindent\begin{minipage}{\linewidth}
\Centering
\begin{tabular}{cc}
\toprule
Spalte 1 & Spalte 2 \\
\midrule
eins & zwei\\
drei & vier\\
\bottomrule
\end{tabular}
\end{minipage}
%
%\begin{minipage}{0.5\linewidth}
\begin{lstlisting}
\usepackage{booktabs}
\begin{tabular}{cc}
	\toprule
		Spalte 1 & Spalte 2 \\
	\midrule
		eins & zwei\\
		drei & vier\\
	\bottomrule
\end{tabular}
\end{lstlisting}
%\end{minipage}


\vspace{-1\baselineskip}
\subsubsection*{Befehls-Übersicht}
\negAbstand

\begin{lstlisting}
\toprule[LEN](			% LEN: Liniendicke
\midrule[LEN]
\bottomrule[LEN]
\cmidrule[LEN](TRIM){A-B}	% Zusätzliche Linie zwischen den Spalten A und B. TRIM: r, r{LEN}, l oder l{LEN}
\addlinespace[LEN]	% Zusätzlicher Zeilenabstand
\end{lstlisting}

\Paket{booktabs} ist kombinierbar mit den Umgebungen \Umgebung{tabular}, \Umgebung{array}(\verb|\usepackage{array}|) und \Umgebung{longtable} (\verb|\usepackage{longtable}|).

%%%%%%%%%%%%%%%%%%%%%
%\columnbreak
\subsectionstage[2]{Feste Tab.-breite (\Paket{tabularx})}
Beispiel (noch näher erläutern!):
\begin{lstlisting}
\usepackage{tabularx}
	\newcolumntype{L}[1]{>{\RaggedRight \arraybackslash}p{#1}} % linksbündig	%mit tabularx neue Spaltentypen definieren.
	\newcolumntype{M}[1]{>{\RaggedRight \arraybackslash}m{#1}} % linksbündig
	
	\newcolumntype{C}[1]{>{\Centering \arraybackslash}p{#1}} % zentriert
	\newcolumntype{D}[1]{>{\Centering \arraybackslash}m{#1}} % zentriert
		
	\newcolumntype{R}[1]{>{\RaggedLeft \arraybackslash}p{#1}} % rechtsbündig
	
	\newcolumntype{B}[1]{>{\arraybackslash}p{#1}}	%Blocksatz
\end{lstlisting}

\negAbstand

%%%%%%%%%%%%%%%%%%%%%
\subsectionstage[4]{Farbige Tab. (\Paket{colortbl})}
\negAbstand
\begin{lstlisting}
\usepackage{array,color} % oder xcolor
\usepackage{colortbl}
\end{lstlisting}

\negAbstand\negAbstand

%%%%%
\begin{multicols}{2}
%
\begin{lstlisting}
\arrayrulecolor{red}		% Farbe der Tabellen-Linien

\begin{tabular}{|
	>{\columncolor{lime}}l|
	>{\columncolor{lightgray}}c|
	}
	\toprule %o.\hline
		eins & zwei\\
		drei & vier\\
	\bottomrule
\end{tabular}
\end{lstlisting}

\arrayrulecolor{red}		% Farbe der Tabellen-Linien
\begin{tabular}{|
	>{\columncolor{lime}}l|
	>{\columncolor{lightgray}}c|
	}
\toprule
eins & zwei\\
drei & vier\\
\bottomrule
\end{tabular}

\columnbreak

\begin{lstlisting}
\begin{tabular}{lc}
	\rowcolor{Goldenrod}
	eins 
	& zwei\\
	drei 
	& \cellcolor{Tan} vier\\
\end{tabular}
\end{lstlisting}

\arrayrulecolor{black}
\begin{tabular}{lc}
\toprule
\rowcolor{Goldenrod}eins & zwei\\
drei & \cellcolor{Tan}vier\\
\bottomrule
\end{tabular}


\end{multicols}
%%%%%

%%%%%%%%%%%%%%%%%%%%%
\subsectionstage[3]{Mehrseitige Tab. (\Paket{longtable})\hspace{-0.8ex}}
Benutzbar wie die \Umgebung{tabular}-Umgebung.

%\begin{multicols}{2}
\begin{lstlisting}
\usepackage{longtable}
\usepackage{booktabs}		% in diesem Beispiel verwendet

\begin{longtable}{c|c|c|}
	\caption{Nährstofftabelle}\\
	% % Kopfzeile der ersten Seite
	\toprule
	...
	\\ \midrule \endfirsthead 
	%% Kopfzeile ab der zweiten Seite
	\toprule
	...
	\\ \midrule \endhead
	% % Fußzeile der ersten Seite
	...
	\bottomrule	\endfoot
	%% Fußzeile ab der zweiten Seite
	...
	\bottomrule \endlastfoot
	<Inhalt der Tabelle>
\end{longtable}
\end{lstlisting}
%\end{multicols}
\negAbstand


%%%%%%%%%%%%%%%%%%%%%
\subsectionstage[2]{Doppelte Linien in Tabellen (\Paket{hhline})}
%\negAbstand


\noindent\begin{minipage}{\linewidth}
\Centering
\begin{tabular}{||cc||c|c||} \hhline{|t:==:t:==:t|}
a & b & c & d \\ \hhline{|:==:|~|~||}
w & x & y & z \\ \hhline{|b:==:b:==:b|}
\end{tabular}
\end{minipage}
%
%\noindent\begin{minipage}{0.5\linewidth}
\begin{lstlisting}
\begin{tabular}{||cc||c|c||} \hhline{|t:==:t:==:t|}
a & b & c & d \\ \hhline{|:==:|~|~||}
w & x & y & z \\ \hhline{|b:==:b:==:b|}
\end{tabular}
\end{lstlisting}
%\end{minipage}
%
\hrule
%
\posAbstand

\noindent\begin{minipage}{\linewidth}
\Centering
\begin{tabular}{||cc||c|c||} \hhline{|t:==:t:==:t|}
a & b & c & d \\\hhline{|:==:|~|~||}
1 & 2 & 3 & 4 \\\hhline{#==#~|=#}
i & j & k & l \\\hhline{||--||--||}
w & x & y & z \\\hhline{|b:==:b:==:b|}
\end{tabular}
\end{minipage}
%
%\begin{minipage}{0.5\linewidth}
\begin{lstlisting}
\begin{tabular}{||cc||c|c||} \hhline{|t:==:t:==:t|}
a & b & c & d \\ \hhline{|:==:|~|~||}
1 & 2 & 3 & 4 \\ \hhline{#==#~|=#}
i & j & k & l \\ \hhline{||--||--||}
w & x & y & z \\ \hhline{|b:==:b:==:b|}
\end{tabular}
\end{lstlisting}
%\end{minipage}
%


\negAbstand

%%%%%%%%%%%%%%%%%%%%%
\subsectionstage[1]{Tabellenformat anpassen}
\negAbstand
\begin{lstlisting}
\setlength{\extrarowheight}{4pt} % extra Zeilenhöhe (Tabellenzeilen) 
\renewcommand{\arraystretch}{Faktor}	% Abstand zwischen Zeilen ändern. (Textzeilen) oder \setstretch{Faktor} (Paket setspace)
\setlength{\tabcolsep}{6pt}		% Abstand zwischen Spalten ändern
\end{lstlisting}
\negAbstand
\hrule \vspace{0.5\baselineskip}
%----- EOF -----