%%%%%%%%%%%%%%%%%%%
\mycolumnbreak
\sectionstage[1]{Gleitumgebungen}

	Gleitende Umgebungen: figure, table.
	Mit h, t, b, p kann die bevorzugte Ausgabe-Stelle angegeben werden.
\begin{lstlisting}
\begin{figure}[htbp]	% h(ier), t(op), b(ottom), p(age)
	\dots
\end{figure}
\end{lstlisting}

\subsubsection{Weitere Befehle}
\begin{lstlisting}
\clearpage		% vor \clearpage werden alle Gleitumgebungen ausgegeben und dann eine neue Seite begonnen.
\pagebreak		% Es wird eine neue Seite begonnen (kein Einfluss auf Gleitumgebungen). 
\end{lstlisting}
\todo{caption, label}

\subsection{Paket \Paket{placeins}}
\begin{lstlisting}
\usepackage[above, below]{placeins} 
\FloatBarrier		% Alle Gleitumgebungen müssen vor \FloatBarrier ausgegeben werden.
% Mit den Optionen above, below darf die Gleitumgebung auch davor (danach) ausgegeben werden, sofern es sich um die selbe Seite handelt.
\end{lstlisting}

\subsection{Paket \Paket{caption}}
Aussehen \& Formatierung anpassen

subpcaption: Bilder nebeneinander (siehe \ref{sec:bilder_neben})

\hrule \vspace{0.5\baselineskip}
%----- EOF -----