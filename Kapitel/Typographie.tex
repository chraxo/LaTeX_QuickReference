%%%%%%%%%%%%%%%%%%
%\columnbreak ~ 
\columnbreak
\sectionstage[2]{Typographie}
Vertiefende Informationen: 
\cite{typokurz}; 
\cite{Struckmann};
\url{http://www.matthiaspospiech.de/latex/dokumentation/typographie/}
\vspace{-0.5\baselineskip}

%%%
\subsection{Typographie \texorpdfstring{(vgl.\cite{Struckmann})}{}} \label{sec:typo}

\subsubsection*{Leerzeichen}

\begin{tabbing}
\hspace{1.7em}	\= \hspace{2.2em}  	\=  \kill  % Musterzeile
%
	xx					\> \lstinline|xx|			\> kein Leerzeichen   \\
	x\,x				\> \lstinline|x\,x|		\>	halbes Leerzeichen	\\
	x x					\> \lstinline|x x|   	\>	(ganzes) Leerzeichen	\\
	x~x					\> \lstinline|x~x|		\> 	geschütztes (ganzes) Leerzeichen 
	\\ \>\> $\rightarrow$ Umbruch verhindert 
\end{tabbing}


\addtolength{\leftmargini}{-10pt}

\begin{itemize}
	\item Abkürzungen: Halbes Leerzeichen. {z.\,B.~~~~\lstinline|z.\,B.|} \\
		Auch möglich: \lstinline|\newcommand{\zB} {\mbox{z.\,B.}\xspace}| (Befehl \verb|\xspace| mit Paket \Paket{xspace} importieren).
	\item Einheiten: Trennung verhindern $\rightarrow$ Geschütztes Leerzeichen ~~~\lstinline|5~kg|.
\end{itemize}
\negAbstand

\addtolength{\leftmargini}{+10pt}

\subsubsection*{Anführungszeichen}
	
\begin{tabbing}
	\hspace{5em} 			\= \kill
%
	\glqq\dots\grqq		\> \lstinline|\glqq\dots\grqq| 	\\
										\> \emph{oder:} \lstinline|"`\dots"'| \\
	\frqq\dots\flqq		\> \lstinline|\frqq\dots\flqq|	\\
	\glq\dots\grq			\> \lstinline|\glq\dots\grq|   	\\
	\frq\dots\flq 		\> \lstinline|\frq\dots\flq| 		\\
	``\dots''					\> \lstinline|``\dots''|  			\\
	`\dots' 					\> \lstinline|`\dots'| 
\end{tabbing}



\subsubsection*{Binde- und andere Striche}
\begin{tabular}{L{1ex}L{4em}L{11em}}
%\hspace{1em} \= \lstinline|\textthreequartersemdash{}| \hspace{-3em} \=  \kill
-			&  \lstinline|-| &  		Divis (Bindestrich)
\\
--		&  \lstinline|--| &  		Halbgeviertstrich (Gedankenstrich)
\\
---		&  \lstinline|---| &  	Englischer Gedankenstrich
\\
\texttwelveudash{} &  \ \befehlsformat{text twelve udash\{\}}  &  Zweidrittel-Geviertstrich
\\
\textthreequartersemdash{} 
&  \ \befehlsformat{text three quarter semdash\{\} }
&   Dreiviertel-Geviertstrich
\end{tabular}

%
\begin{description}
	\item[Bindestrich] \lstinline|-|	\\
SOS-Ruf ~~~ \lstinline|SOS-Ruf|	\\
ISBN 978-3-642-12880-6

\item[Gedankenstrich] \lstinline|--| \\
	Er kam -- und ging gleich wieder.\\
	\lstinline|Er kam -- und ging gleich wieder.|
\end{description}

Worttrennung: siehe Abschnitt \ref{sec:trennung}

\setlength{\columnseprule}{0.5pt}
\hrule \vspace{0.5\baselineskip}
%----- EOF -----