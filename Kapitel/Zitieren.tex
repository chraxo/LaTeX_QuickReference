
\columnbreak
\section[Zitieren \& Literaturverzeichnis]{Zitieren \& Literaturverz. \stage{3}}

\lstinline|\cite[S.\,36]{Kuchling2007}| \linebreak $\Rightarrow$[Kuch,~S.36]

\noindent\lstinline|\nocite{*}	% Alle Quellen (auch nicht zitierte} in's Literaturverzeichnis eintragen.|

\subsectionstage[4]{manuell}
\negAbstand
\begin{lstlisting}
\begin{thebibliography}{LK} % Statt LK das Längste Kürzel (z.B. "Kuch~07") eintragen.
\bibitem[Jahn~12]{Jahne2012}
	B.\,Jähne.
	\newblock \textit{Digitale Bildverarbeitung}.
	\newblock Springer, Berlin, Heidelberg, 2012.
...
\end{thebibliography}
\end{lstlisting}


\subsubsectionstage[2]{Empfehlung Zitierformat für Bc.- und Mst.-Arbeiten}

	[Jahn~12]
	B.\,Jähne.
	\newblock {\itshape Digitale Bildverarbeitung}.
	\newblock Springer, Berlin, Heidelberg, 2012.\\[1ex]
	%
	[Wiki~13a]	Wikipedia. 
	\newblock Artikel {\itshape Bayes-Klassifikator}.
	\newblock \url{http://de.wikipedia.org/wiki/Bayes-Klassifikator},
	\newblock Abgerufen am 02.02.2014.
\negAbstand

\subsection{Literaturverzeichnis mit BibTeX}
Für die Erstellung von .bib-Files empfohlenes Programm: JabRef \newline
\lstinline|\bibliographystyle{wmaainf}| \newline
\lstinline|\bibliography{bibfile.bib}|
\negAbstand

\subsubsection*{Beispiel für \nolinkurl{.bib}-Datei}
\negAbstand
\begin{lstlisting}
@BOOK{Kuchling2007,
	title			={Taschenbuch der Physik},
	publisher	={Hanser},
	year			={2007},
	author		={Horst Kuchling},
	series		={19.\,Auflage},
	owner			={ca251},
	timestamp	={2011.08.12},
}
\end{lstlisting}
%
Mögliche Literatur-Typen: @article, @book, @booklet, @conference, @inbook, @incollection, @inproceedings, @manual, @mastersthesis, @misc, @phdthesis, @proceedings, @techreport, @unpublished

\subsubsection*{Literaturstile für BibTeX}
{\small
\begin{itemize}
	\item wmaainf.bst: Stile plaindin, unsrtdin, alphadin, abbrvdin
	\item natbib
	\item natbib + Paket \Paket{custom-bib} (s.a. \url{http://www.golatex.de/brauchbarer-bibtex-stil-t2163.html})
	\item \Paket{babelbib}: Sprachumschaltung für Bib-Stile
\end{itemize}
}

\subsectionstage[0]{Paket \Paket{biblatex}}
\textcolor{red}{\footnotesize
This Chapter is still missing. The package ``biblatex'' offers many functions.
}

\hrule \vspace{0.5\baselineskip}
%----- EOF -----