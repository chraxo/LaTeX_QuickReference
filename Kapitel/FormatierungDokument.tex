\pagebreak

\sectionstage[2]{Formatierung des Dokuments}
\label{cha:DokFormatierung}
Hinweis: Diese Zusammenfassung geht von der Verwendung einer der Koma-Script-Dokumentklasse aus.

\subsectionstage[0]{Seitenlayout}
\label{cha:Seitenlayout}
\textcolor{red}{fehlt (noch)}

\noindent Notizen:
\begin{itemize}
	\item siehe auch \todo{\cite[S.\,579]{Braune}}
	\item raggedbottom, flushbottom, raggedcolumns, flushcolumns
	\item weitere geometry-Optionen: bindingoffset, inner, outer
\end{itemize}

\notizenplatz\notizenplatz

\subsectionstage[1]{Dokumenten-Aufbau}
\begin{lstlisting}
\documentclass [KLASSENOPTIONEN] {DOKUMENTKLASSE}
>HEADER<
\begin{document}
	Inhalt...
\end{document}
\end{lstlisting}

%----- ------
\subsectionstage[3]{Klassenoptionen}
%\renewcommand{\arraystretch}{1.5} %Abstand zwischen Zeilen ändern
(angelehnt an \cite[S.\,9]{l2kurz}) \newline
\begin{tabular}{L{0.2\linewidth}L{0.65\linewidth}}
\toprule
	10pt 11pt 12pt
	& Schriftgröße (Standard ist 10 pt)
	\\ \midrule
	a4paper a5paper
	& Papierformat
	\\ \midrule
	fleqn
	& Gleichungen linksbündig statt zentriert.
	\\ \midrule
	leqno
	& Gleichungsnummer links statt rechts
	\\ \midrule
	titlepage
	notitlepage
	& Eigene Seite für Titel und Zusammenfassung? (Standard für report und book: titlepage).
	\\ \midrule
onecolumn \linebreak
twocolumn
& für ein- oder zweispaltiges Dokument. Die Voreinstellung ist immer onecolumn.
	\\ \midrule
oneside \linebreak
twoside
& Seitengestaltung für ein- oder zweiseitigen Druck. (Standard: oneside; bei book: twoside)
\\ \midrule
DIV=calc	& Satzspiegel automatisch berechnen (statt ``calc'' auch ..,6,7,8,9,10,..)
\\ \midrule
BCOR=1cm	&~~~Bindekorrektur
\\ \midrule
headings=small &~~~~~~ kleinere Überschriften
\\ \bottomrule
\end{tabular}

\bigskip\bigskip

%----- ------
\noindent\begin{minipage}{\linewidth}
\subsectionstage[2]{Dokumentklassen}
Tabelle (entnommen aus \cite[S.\,8\,f]{l2kurz})

\begin{tabular}{L{9ex}L{28ex}}
\toprule
	article	
	& für Artikel in wissenschaftlichen Zeitschriften, kürzere 	Berichte u.v.a.
\\
	report
	&für längere Berichte, die aus mehreren Kapiteln bestehen (Diplomarbeiten, Dissertationen u.ä.)
\\
	book
	&für Bücher
\\
	scrartcl, scrreprt, scrbook
	& {\gray(empfohlen)} Die sog. KOMA-Klassen (vgl.\,\cite{scrguide}) sind Varianten der o.\,g. Klassen mit besserer Anpassung an DIN-Papierformate und „europäische“ Typographie.
	Zudem erlauben sie mittels Optionen die Anpassung des Layouts.
\\
	beamer
	& für Präsentationen
\\ \bottomrule
\end{tabular}
\end{minipage}

\subsubsectionstage[0]{Schriftarten}
\label{cha:Schriftarten}
\textcolor{red}{fehlt (noch)}
\notizenplatz\notizenplatz


% ----- Zeilenabstand -----
\subsectionstage[3]{Zeilenabstand (\Paket{setspace})}
\negAbstand

\begin{lstlisting}
\usepackage[OPT]{setspace}	% OPT: (doublespacing, onehalfspacing, singlespacing)
\doublespacing	% Ab jetzt Zeilenabstand doppelt
\onehalfspacing	
\singlespacing	
\setstretch{1.8}	% 1.8-facher SA.
\begin{spacing}{1.8} ... \end{spacing}
\end{lstlisting}

\subsectionstage[1]{Absätze}
\label{cha:Absatz}
%
\paragraph*{Absatzeinrückung} ~\linebreak
\lstinline|\parindent 0pt		% Einrückungslänge einstellen.|
	{\footnotesize\todo{empfohlene Variante?}}%
%
\paragraph*{Absatzabstände} ~\linebreak
\lstinline|\KOMAoptions{parskip=half}| \linebreak
Weitere Werte für parskip:
\begin{description}
	\item[full] vertikaler Abstand + Leerraum (mindestens \lstinline|1em|) am Ende der letzten Absatz-Zeile
	\item[full--] vertikaler Abstand
	\item[full+] vertikaler Abstand + Leerraum (mindestens \lstinline|0.33\baselineskip|)
	\item[full*] vertikaler Abstand + Leerraum (mindestens \lstinline|0.25\baselineskip|)
	\item[half] vertikaler Abstand (halbe Zeile)
	\item[never] (selbsterklärend)
	\item[...] (weitere sind hier nicht aufgeführt)
\end{description}

% -----  -----
\subsectionstage[1]{Formatierung der Überschriften ändern}
\label{cha:ChapFormat}
Überschriften-Farben ändern (nur mit Komascript-Dokumentklasse):
\begin{lstlisting}
\addtokomafont{section}{\color[HTML]{841010}}
\addtokomafont{subsection}{\color[HTML]{844A10}}
\addtokomafont{subsubsection}{\color[HTML]{848410}}
\end{lstlisting}

(siehe \scite{scrguide}{56f})




%\negAbstand
%\begin{lstlisting}
%\usepackage{titlesec}		%% ??? Fehler!
%\titleformat{\section}
%{\color{red}\normalfont\Large\bfseries}
%{\color{red}\thesection}{1em}{}
%\end{lstlisting}
siehe auch \url{http://texblog.org/tag/titlesec/}



% ----- Seitenspiegel / Seitenränder -----
%\columnbreak
\subsection{Satzspiegel / Seitenränder}


%----- -----
\subsubsectionstage[4]{mittels DIV}
Typographisch günstigste Methode: Dokumentclass-Option {\green \lstinline|DIV=calc|}.
Wird eine Zahl gesetzt ({z.\,B.\,\green\lstinline|DIV=9|}), kann der Seitenrand beeinflusst werden.


%----- -----
\subsubsectionstage[4]{Seitenränder mit \Paket{geometry}}
\negAbstand
\begin{lstlisting}
\usepackage[
	left=1.6cm, right=1cm,
	top=0.5cm, bottom=0cm, 
	includefoot,	% Fußzeile NICHT ignorieren
	includehead,	% Kopfzeile NICHT ignorieren
	%includeheadfoot	% includefoot + includehead
]{geometry} 
\end{lstlisting}
Werden manche der Werte (left, right, top, bottom) weggelassen, so wird der Rest automatisch bestimmt.

%----- -----
\subsubsectionstage[2]{Überblick: Layout-Variablen}\label{sec:layout_var}

\footnotesize
\addtolength{\leftmargini}{-10pt}
\begin{itemize}
	\item Die obere Referenzlinie hat einen Abstand von 1 inch zum oberen Seitenrand
	\item Die linke Referenzlinie hat einen Abstand von 1 inch zum linken Seitenrand
	\item \lstinline|\topmargin|: Abstand zwischen der oberen Referenzlinie und der Oberkante des Headers
	\item \lstinline|\headheight|: Höhe des Headers
	\item \lstinline|\headsep|: Abstand zwischen Unterkante des Headers und dem Body
	\item \lstinline|\textheight|: Höhe des Body
	\item \lstinline|\footskip|: Abstand von der Unterkante des Body bis zur \emph{Unterkante} des Footers
	\item \lstinline|\oddsidemargin|: Abstand zwischen der linken Referenzlinie und linken Kante des Headers
	\item \lstinline|\textwidth|: Breite des Body
	\item \lstinline|\marginparsep|: Abstand von der rechten Seite des Body bis zur linken Kante der Margin notes
	\item \lstinline|\marginparwidth|: Breite der Margin notes
	\item \lstinline|\marginparpush|: Abstand zwischen Unterkante einer Margin note und der Oberkante der nächsten Margin note
\end{itemize}
\addtolength{\leftmargini}{10pt}

\normalsize

	Im Normalfall sollten diese Variablen \textbf{nur gelesen}, aber \textbf{nicht geändert} werden.
	Zum Einstellen der Seitenränder sollte zum Beispiel das Paket \Paket{geometry} verwendet werden.

Bei mehrspaltigem Layout ist die Variable \lstinline|\linewidth| anstatt \lstinline|\textwidth| nützlich, die die Breite der Zeile (ungefähr die Spaltenbreite) angibt.

Tipp: Mit \lstinline|\enlargethispage{1\baselineskip}| einmalig die textheight der aktuellen Seite um 1~Zeile vergrößern.

\subsectionstage[2]{Anhang}
\begin{lstlisting}
\appendix
\renewcommand{\thechapter}{\Alph{chapter}}  % römische Kapitelnummerierung

\makeatletter
\@addtoreset{figure}{section}
\@addtoreset{table}{section}
\makeatother

\renewcommand{\thefigure}{\thesection.\arabic{figure}}
\renewcommand{\thetable}{\thesection.\arabic{table}}
\end{lstlisting}
\notizenplatz

% ----- Kopf-/Fußzeilen -----

\subsectionstage[4]{Seitennummerierung}
\negAbstand

\begin{tabbing}
\lstinline|\pagenumbering{arabic}| 
\hspace{0.7ex}	\=  
\hspace{1em}, \= \hspace{1em}, \= \hspace{1em}, \= \hspace{1em}, \= \hspace{1em}, \=\hspace{1em} \kill

\lstinline|\pagenumbering{arabic}|	\> 1, \> 2, \> 3, \> 4, \> \dots 
\\
\lstinline|\pagenumbering{Roman}|		\> I, \> II, \> III, \> IV,  \> \dots
\\
\lstinline|\pagenumbering{roman}|		\> i, \> ii, \> iii, \> iv,  \> \dots 
\\
\lstinline|\pagenumbering{Alph}| 		\> A, \> B, \> C, \> D,  \> \dots
\\
\lstinline|\pagenumbering{alph}|		\> a, \> b, \> c, \> d,  \> \dots 
\end{tabbing}
%%%%%%%%%%%%%%%%%%%%%%%%%%%%%%%%%
\columnbreak
\subsectionstage[3]{Kopf-/ Fußzeilen (\Paket{scrpage2})}
\negAbstand

\begin{lstlisting}
\usepackage{scrlayer-scrpage}
\pagestyle{scrheadings}	% Auf den neuen Seitenstil umschalten
		% Auf der ersten Seite eines Kapitels wird i.d.R. automatisch auf den Seitenstil scrplain umgeschaltet. Auf der nächsten Seite gilt dann wieder scrheadings.
\end{lstlisting}

\noindent\begin{minipage}{\linewidth}

\subsubsectionstage[4]{Löschen der Stile (um "`frisch"' anzufangen)}
\negAbstand

\begin{lstlisting}
 \clearscrheadfoot	% Beide Stile (scrheadings, scrplain) komplett leeren.
 \clearscrheadings	% Stil scrheadings komplett leeren
 \clearscrplain			% Stil scrplain komplett leeren
\end{lstlisting}
\end{minipage}
\negAbstand

\subsubsectionstage[3]{Definition}
\negAbstand

\begin{lstlisting}
\ihead[Text für scrplain-Stil] 
		{Text für scrheadings-Stil}
\ihead{Kopfzeile Innen}
\chead{Kopfzeile Mitte}
\ohead{Kopfzeile Außen}
\ifoot{Fußzeile Innen}
\cfoot{Fußzeile Mitte}
\ofoot{Fußzeile Außen}
\end{lstlisting}

%\negAbstand
%\begin{minipage}{0.7\linewidth}
%\centering
%\fbox{\includegraphics[width=\linewidth]{Bilder/scrguide_S216_scrpage2Befehle}}
%\end{minipage}

\posAbstand
\subsubsectionstage[3]{Befehle für automatischen Text\hspace{-0.5ex}}\vspace{-0.6\baselineskip}
\begin{lstlisting}
\pagemark		% Seitenzahl.
\thepage		% Seitenzahl. (Alternative).
\automark[<rechteSeite>]{<linkeSeite>}	% Definiert \rightmark und \leftmark. (chapter, section, subsection, ...)
\rightmark	% z.B. Kapitel-Name incl.Nr.
\leftmark		% z.B. Section-Name incl.Nr. (\leftmark nur bei Dokumentoption "`twoside"' verfügbar)
\headmark		% 
\manualmark	% 
\end{lstlisting}

\subsubsectionstage[3]{Kapitel-Nummerierung ausschalten}
\label{cha:KapNumAus}
\todo{Quelle}
\negAbstand
\begin{lstlisting}
\renewcommand*{\chaptermarkformat}{} 
\renewcommand*{\sectionmarkformat}{}
\renewcommand*{\subsectionmarkformat}{}

% Orignalwerte  (Nummerierung eingeschaltet): 
\renewcommand{\chaptermarkformat} {\chapappifchapterprefix{\ }\thechapter\autodot\enskip}
\renewcommand*{\sectionmarkformat} {\thesection\autodot\enskip}
\renewcommand*{\subsectionmarkformat} {\thesubsection\autodot\enskip}
\end{lstlisting}

\subsubsectionstage[3]{Schriftformatierung}
\negAbstand
\begin{lstlisting}
\renewcommand*{\headfont} {\normalfont \bfseries}	% Schriftart für Kopf- und Fußzeile
\renewcommand*{\footfont} {\normalfont \itshape} % Von der Kopfzeile abweichende Schriftart für die Fußzeile.
\renewcommand*{\pnumfont} {\normalfont} % Von headfont und footfont abweichende Formatierung für Seitenzahlen
\end{lstlisting}

\subsubsectionstage[3]{Linien}
\setlength{\columnseprule}{0.5pt}
\negAbstand
\begin{lstlisting}
\setheadsepline[\textwidth]{.4pt} % Linie unter der Kopfzeile
\setfootsepline{.4pt} 	% Linie über der Fußnote

\setheadtopline{2pt}[\color{red}]	% Linie über der Kopfzeile
\setfootbotline{2pt}		% Linie unter der Fußnote
\end{lstlisting}

\bigskip
\myhrule
\bigskip

% ----- Mehrspaltiger Textsatz -----
%%%%%%%%%%%%%%%%%%%
\setlength{\columnseprule}{0.5pt}

\columnbreak
\subsectionstage[3]{Mehrspaltiger Textsatz (\Paket{multicol})}
\negAbstand

\begin{lstlisting} 
\begin{multicols}{N}		% multicols*:ohne automatischen Spaltenausgleich; N: Anzahl Spalten
	\columnbreak					% manueller Spaltenumbruch
	\hrule								% Horizontale Linie
\end{multicols}
\end{lstlisting}

\subsubsection*{Einstellungen}
\negAbstand

\begin{lstlisting} 
\setlength{\columnseprule}{0.5pt}	% Trennlinie: Strichstärke (optional)
\setlength{\columnsep}{10pt}			% Abstand der Spalten
\end{lstlisting}

\begin{description}

\item[Hinweis] \textcolor{red}{Floats (Gleitumbegungen) sind innerhalb von multicols nicht erlaubt.} 

\item[Hinweis] Werden zwei multicols-Umgebungen ineinander verschachtelt, so wird die Innere Umgebung durch Spaltenumbruch auf die äußere verteilt.

\item[Hinweis] Verschachtelte multicols-Umgebungen können Probleme beim Umbruch bereiten. In diesem Fall sollten die inneren multicols-Umgebungen in mehrere kleinere multicols-Umgebungen aufgesplittet werden.
\end{description}


% ----- Querformat -----
\noindent\begin{minipage}{\linewidth}
\subsectionstage[3]{Querformat (einzelne Seiten)}
\negAbstand

\begin{lstlisting}
\usepackage{pdflscape} 	% Querformat wird in PDF-Ausgabe gedreht dargestellt.
\usepackage{lscape}			% Querformat bleibt in PDF-Ausgabe aufrecht
\begin{landscape} ... \end{landscape}
\end{lstlisting}
\end{minipage}
%
Komplettes Dokument im Querformat: Dokumentoption "`landscape"'.

\subsectionstage[3]{PDF-Ausgabe \& \Paket{hyperref}}
\Paket{hyperref} sollte als letztes aller Pakete eingebunden werden (weitere Infos unter \url{http://de.wikibooks.org/wiki/LaTeX-W%C3%B6rterbuch:_hyperref}).
\begin{lstlisting}
\usepackage[pdftex,
	hidelinks,
	pdfauthor=	{cax},	% Autorname
	pdfkeywords={},			% Schluesselwoerter
	pdfdisplaydoctitle=true, 
	pdftitle=		{},			% Titel
	pdfsubject=	{}, 		% Thema
	%colorlinks=false,
	%linkcolor=MidnightBlue,
	%urlcolor=black,
	%citecolor=black,
	%%%
	bookmarksopen=false,
	%pdffitwindow=true,
	pdfpagelayout={SinglePage},	% (SinglePage, OneColumn, TwoColumnLeft, TwoColumnRight, TwoPageLeft, TwoPageRight)
	pdfpagemode={UseNone},	% (UseNone, UseThumbs, UseOutlines, FullScreen, UseOC, UseAttachments)
	%pagebackref 	=true, 	%zurück-Button im Acrobat-Reader
]{hyperref} 
\end{lstlisting}

\notizenplatz
\notizenplatz

\hrule \vspace{0.5\baselineskip}
%----- EOF -----
