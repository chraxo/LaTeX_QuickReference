%%%%%%%%%%%%%%%%%%%

\columnbreak
\sectionstage[2]{Programmieren}

\subsectionstage[2]{Variablen}


\subsubsectionstage[2]{Counter}
\negAbstand
\begin{lstlisting}
\newcounter{MYCTR}[resetCtr]
\setcounter{MYCTR}{3} % setzt MYCTR auf 3
\addtocounter{MYCTR}{3} % addiert 3
\stepcounter{MYCTR}	% +1

\value{MYCTR}				% Programmierausgabe
\themyctr						% Textausgabe
\Roman{MYCTR}	% " anderes Zahlenformat 
(\arabic, \alph, \Alph, \roman, \Roman)
\end{lstlisting}
\LaTeX-Standard-Counter: part, chapter, section, subsection, subsubsection, paragraph, subparagraph, \textbf{page} (Seitennummer), equation, figure, table, footnote, mpfootnote, enumi, enumii, enumiii, enumiv (enumerate-Umgebung)

\bigskip


\subsubsectionstage[3]{Längen} \label{cha:Laengen}
In Einheiten angeben (intern in pt): ex, em, pt, mm, cm, in (vgl. Abschnitt \ref{sec:abstaende})
\begin{lstlisting}
\newlength{\MYLEN}
\setlength{\MYLEN}{2em}
\addtolength{\MYLEN}{10pt}
\settowidth{\MYLEN}{TEXT}		% Auf Länge von TEXT setzen.
\settoheight{\MYLEN}{TEXT}	

\setlength{\MYLEN}{VALUE plus 1em minus 1em} % Variable Länge
\the\MYLEN		% Wert von MYLEN ausgeben
\end{lstlisting}


\subsubsectionstage[2]{Booleans}
\begin{lstlisting}
\newboolean{MYBOOL}
\setboolean{MYBOOL}{false}

\boolean{MYBOOL} % Verwendung im Code
\ifthenelse{MYBOOL}{<true code>}{<false code>}
\end{lstlisting}


\subsectionstage[3]{Befehle}
\vspace{-0.5\baselineskip}
\begin{lstlisting}
\newcommand{\NAME}[N]{CODE} 	% optional N: Anzahl Parameter; Im Code werden die Parameter mit #1, #2, usw. eingefügt.
\newcommand{\beispiel}[2][StandardFürErstesArgument]{#1 #2}  % Bis zu 9 Argumente. Erstes Argument kann optional sein.
\renewcommand{\NAME}[N]{CODE}	% Bestehenden Befehl überschreiben
\end{lstlisting}
Beispiel:
\begin{lstlisting}
\newcommand{\meinBefehl}[2] {Erst #1. Dann \textbf{#2} }
\meinBefehl{Eins}{Zwei}
\end{lstlisting}
\newcommand{\meinBefehl}[2] {Erst #1. Dann \textbf{#2}. }
\hspace{2em} \meinBefehl{Eins}{Zwei}

\posAbstand

\subsectionstage[2]{Umgebungen}
\negAbstand
\begin{lstlisting}
\newenvironment{NAME}[N]{CODE_BEGIN} {CODE_ENDE}
\renewenvironment{NAME}[N]{CODE_BEGIN} {CODE_ENDE}
\end{lstlisting}



%%%%%%%%%%%%%%
\subsectionstage[2]{Paket \Paket{calc}}
\negAbstand
In \lstinline|\setcounter|-Befehlen etc. können berechnete Werte verwendet werden. Operatoren: \lstinline|+ - * /|

Erlaubt: "`\lstinline|2cm*4|"', "`\lstinline|2cm+4pt|. Nicht erlaubt: "`\lstinline|4*2cm|"', "`\lstinline|2cm+4|"' (Reihenfolge).

Weitere Befehle: \lstinline| \real{1.6}, \ratio{}{}, \maxof{}{}, \minof{}{}|

\bigskip\bigskip

%\subsection{\titleformat{\red}  Paket ifthen}
%for-Schleife wie?
%\lstinline|\ifthenelse{}{}{}|, \lstinline|\whiledo|
%\bigskip\bigskip


%%%%%%%%%%%%
\subsectionstage[3]{Paket \Paket{xifthen}}
\negAbstand

\begin{lstlisting}
\boolean{<boolean>}
\newboolean{<boolean>}	%<boolean> contents of alphanumeric characters.
\setboolean{<boolean>}{<truth value>}	% <truthvalue> = (true,false)
\ifthenelse{<test expression>}{<true code>}{<false code>}

Test expressions:
<int1> = <int2>
<int1> < <int2>
<int1> > <int2>
\isodd{<number>}
\lengthtest{<dimen1> = <dimen2>}	% Test von Längen
\isundefined <command>
\equal{<string1>}{<string2>}
\end{lstlisting}
%
{\small\ttfamily Weitere <booleans>: mmode (Are we in math mode?), hmode (Are we in horizontal mode?), vmode (Are we in vertical mode?), etc.
}
%
\begin{lstlisting}
<expression1> \AND <expression2>
<expression1> \OR <expression2>
\NOT <expression>
\(<expression>\)

usw.

\newtest{<command>}[<n>]{<test expression>}
\end{lstlisting}

Weitere Details in der Paketdoku.

\bigskip
% ----- -----
\subsection{Code-Snippets}

\subsubsection[FOR-Schleife]{FOR-Schleife \mytitleformat{(vgl.\cite{Posp08})}}
\negAbstand
\begin{lstlisting}
%\forloop{zaehlvariable}{von}{bis}{schleifeninhalt}
\newcommand{\forloop}[4]{
	\newcounter{#1}
	\setcounter{#1}{#2}
	\whiledo { \NOT {\value{#1} > #3} }
	{
		#4
	\stepcounter {#1} %Schleifenzähler erhöhen
	}
}
% Bsp:
\forloop{loopii}{1}{5}{i=\theloopii}
\end{lstlisting}
Alternative: Paket \Paket{forloop} verwenden.

\subsubsection[switch-case]{switch-case (vgl.\cite{Posp08})}
\negAbstand
\begin{lstlisting}
\newcommand{\myCase}[1]{
	\ifcase #1 	% #1 == 0
		Antwort 0
	\or 				% #1 == 1
		Antwort 1
	\or 				% #1 == 2
		Antwort 2
	\else				% #1 > 2
		else
	\fi
}
\myCase{2} % Textausgabe: "Antwort 2"
\end{lstlisting}
\subsubsection[Code wiederholen]{Code wiederholen (vgl.\cite{Posp08})}
\negAbstand
\begin{lstlisting}
%----- \repeatcode{N}{CODE} ------
\newcounter{repeatCounter}
\newcommand{\repeatcode}[2]{ 
	\setcounter{repeatCounter}{1}
	\whiledo { \NOT {\value{repeatCounter} > #1} }
	{%
		#2%
		\stepcounter{repeatCounter}%
	}
}	
% Bsp.: \repeat{3}{Test}	% ergibt TestTestTest
\end{lstlisting}

\hrule \vspace{0.5\baselineskip}
%----- EOF -----
