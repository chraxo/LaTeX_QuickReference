\subsectionstage[3]{Schriftformatierung}

\subsubsection*{Schriftgröße \stage{3}}
\negAbstand
\ifthenelse{\boolean{a5}}{}{\begin{multicols}{2}}
\begin{tabbing}
		\lstinline|\normalsize|	\hspace{1em}	\=  \kill
		%
		\tiny{tiny}						\> \lstinline|\tiny| 				 	\\
		\scriptsize{scriptsize}			\> \lstinline|\scriptsize|   	\\
		\footnotesize{footnotesize}	\> \lstinline|\footnotesize|	\\
		\small{small}					\> \lstinline|\small| 			  \\
		\normalsize{normalsize}			\> \lstinline|\normalsize| 	  \\
		\large{large}								\> \lstinline|\large| 			  \\
		\Large{Large}								\> \lstinline|\Large| 			  \\
		\LARGE{LARGE}								\> \lstinline|\LARGE| 			 	\\
		\huge{huge}									\> \lstinline|\huge| 				  \\
		\Huge{Huge}									\> \lstinline|\Huge| 					\\
\end{tabbing}

\vspace{-2\baselineskip}
%
\subsubsection*{Schriftstile \stage{3}}
\negAbstand
\begin{tabbing}
		\hspace{6.0em}	\= textnormal textx \hspace{0em} \=  \kill
		%
	\emph{Her\emph{vor}hebung}	
		\> \lstinline|\emph{Her\emph{vor}hebung}|
		\\
	\textrm{Antiqua} 	
		\> \lstinline|\textrm{Serif}|
		\> \lstinline|\rmfamily|		
		\\
	\textsf{Serifenlose} 		
		\> \lstinline|\textsf{Grotesk}|
		\> \lstinline|\sffamily| 
		\\
	\texttt{Monospaced} 	
		\> \lstinline|\texttt{Mono}|	
		\> \lstinline|\ttfamily| 			
		\\[1ex]
	\textmd{normal} 		
		\> \lstinline|\textmd{n.fett}|
		\> \lstinline|\mdseries|
		\\
	\textbf{fett,\,breiter} 		
		\> \lstinline|\textbf{fett}|
		\> \lstinline|\bfseries|
		\\[1ex]
		%%
	\textup{aufrecht} 		
		\> \lstinline|\textup{aufrecht}|
		\> \lstinline|\upshape| 	
		\\
	\textsl{geneigt} 		
		\> \lstinline|\textsl{geneigt}|
		\> \lstinline|\slshape| 	
		\\
	\textit{kursiv} 		
		\> \lstinline|\textit{kursiv}|
		\> \lstinline|\itshape| 	
		\\
	\textsc{Kapitälchen} 		
		\> \lstinline|\textsc{Kapit.}|
		\> \lstinline|\scshape| 	
		\\
	\textnormal{\parbox{7em}{Grundschrift}} 		
		\> \lstinline|\textnormal{text}|
		\> \lstinline|\normalfont| 	
		
\end{tabbing}
Hinweis: Je nach gewählter Schriftart stehen nicht alle Schriftstile zur Verfügung.

{\small
\begin{tabbing}

fcolorbox\hspace{2ex}	\= \kill% Musterzeile
	
	\fbox{fbox}	\> \lstinline|\fbox{Umrahmung}|
	\\
	\underline{underline} 	\> 
	\lstinline|\underline{}| 
	(ABER: Kein Um- \\ \> bruch möglich!) 
	\\
	
	\textcolor{red}{textcolor} \> \lstinline|\textcolor{red}{Farbiger Text}| 
	\\ \> (vgl.\, Kap.\,\ref{cha:Farben})
	\\
	
	\colorbox{lime}{colorbox} \> \lstinline|\colorbox{lime}{TEXT}|
	\\
		
	\fcolorbox{red}{lightgray}{fcolorbox} \> \lstinline|\fcolorbox{red}{lightgray}{TEXT}|
	
	\\
	hoch\textsuperscript{gestellt}	\> \lstinline|\textsuperscript{TEXT}|
	\\
	tief\textsubscript{gestellt}		\> \lstinline|\textsubscript{TEXT}|
	
\end{tabbing}
}
\noindent(siehe auch Kap.\,\ref{cha:Farben}.\nameref{cha:Farben})

%\vspace{-1\baselineskip}
\setlength{\columnseprule}{0pt}
\begin{multicols}{2}

\vspace{-2\baselineskip}
\subsubsection*{Paket \Paket{ulem}}

\lstinline|\usepackage[normalem]{ulem}|

\begin{tabbing}
	dashulin	\=  \kill % Musterzeile
\uline{uline}					\> \lstinline|\uline{}| \\
\uuline{uuline}				\> \lstinline|\uuline{}| \\
\uwave{uwave}					\> \lstinline|\uwave{}| \\
\sout{sout}						\> \lstinline|\sout{}| \\
\xout{xout}						\> \lstinline|\xout{}| \\
\dashuline{dashuline}	\> \lstinline|\dashuline{}| \\
\dotuline{dotuline}		\> \lstinline|\dotuline{}| \\
\end{tabbing}

\vspace{-2\baselineskip}
\columnbreak
\subsubsection*{Paket \Paket{fancybox}}

\begin{tabbing}
	\hspace{1ex}	\=  \kill % Musterzeile
	\> \lstinline|\usepackage{fancybox}|    \\
	\> \shadowbox{\lstinline|\shadowbox{}|} \\
	\> \doublebox{\lstinline|\doublebox{}|} \\
	\> \ovalbox{\lstinline|\ovalbox{}|} 		\\
	\> \Ovalbox{\lstinline|\Ovalbox{}|} 		\\
\end{tabbing}

\end{multicols}\setlength{\columnseprule}{0.5pt}

\negAbstand
