%%%%%%%%%%%%%%%%%%%
\pagebreak
\sectionstage[3]{Formatierung}
\setlength{\columnseprule}{0.5pt}

\subsectionstage[3]{Überschriften}
\vspace{-0.5\baselineskip}
Dokumentklasse \emph{article}:
\begin{lstlisting}
\section 				{TITEL}
\subsection 		{TITEL}
\subsubsection 	{TITEL}
\end{lstlisting}

\noindent
Dokumentklasse \emph{report} oder \emph{book}:
\begin{lstlisting}
\part {TITEL} % Nur mit Book-Vorlage verfügbar
\chapter 				{TITEL}
\section 				{TITEL}
\subsection 		{TITEL}
\subsubsection 	{TITEL}
\end{lstlisting}
%
Bei Format-Problemen im PDF-TOC:
\begin{lstlisting}
\section{\texorpdfstring{ÜBERSCHRIFT}
{ÜBERSCHRIFT FÜR PDF-TOC}}
\end{lstlisting}

\negspace\negspace

\subsubsection*{Keine Nummerierung}
\noindent\lstinline|\addsec{Anhang}	%Section ohne Nummer|
\linebreak
\lstinline|\section* {TITEL} % keine Nummerierung, kein toc-Eintrag|

\negspace
\subsubsection*{Inhaltsverzeichnis (TOC) beeinflussen}
\negspace
%
\begin{lstlisting}
\section[toc-Eintrag]{TITEL}
\addcontentsline{toc}{Kapitelebene}{TITEL}
\end{lstlisting}


\subsectionstage[3]{Schriftformatierung}

\subsubsection*{Schriftgröße \stage{3}}
\negAbstand
\ifthenelse{\boolean{a5}}{}{\begin{multicols}{2}}
\begin{tabbing}
		\lstinline|\normalsize|	\hspace{1em}	\=  \kill
		%
		\tiny{tiny}						\> \lstinline|\tiny| 				 	\\
		\scriptsize{scriptsize}			\> \lstinline|\scriptsize|   	\\
		\footnotesize{footnotesize}	\> \lstinline|\footnotesize|	\\
		\small{small}					\> \lstinline|\small| 			  \\
		\normalsize{normalsize}			\> \lstinline|\normalsize| 	  \\
		\large{large}								\> \lstinline|\large| 			  \\
		\Large{Large}								\> \lstinline|\Large| 			  \\
		\LARGE{LARGE}								\> \lstinline|\LARGE| 			 	\\
		\huge{huge}									\> \lstinline|\huge| 				  \\
		\Huge{Huge}									\> \lstinline|\Huge| 					\\
\end{tabbing}

\vspace{-2\baselineskip}
%
\subsubsection*{Schriftstile \stage{3}}
\negAbstand
\begin{tabbing}
		\hspace{6.0em}	\= textnormal textx \hspace{0em} \=  \kill
		%
	\emph{Her\emph{vor}hebung}	
		\> \lstinline|\emph{Her\emph{vor}hebung}|
		\\
	\textrm{Antiqua} 	
		\> \lstinline|\textrm{Serif}|
		\> \lstinline|\rmfamily|		
		\\
	\textsf{Serifenlose} 		
		\> \lstinline|\textsf{Grotesk}|
		\> \lstinline|\sffamily| 
		\\
	\texttt{Monospaced} 	
		\> \lstinline|\texttt{Mono}|	
		\> \lstinline|\ttfamily| 			
		\\[1ex]
	\textmd{normal} 		
		\> \lstinline|\textmd{n.fett}|
		\> \lstinline|\mdseries|
		\\
	\textbf{fett,\,breiter} 		
		\> \lstinline|\textbf{fett}|
		\> \lstinline|\bfseries|
		\\[1ex]
		%%
	\textup{aufrecht} 		
		\> \lstinline|\textup{aufrecht}|
		\> \lstinline|\upshape| 	
		\\
	\textsl{geneigt} 		
		\> \lstinline|\textsl{geneigt}|
		\> \lstinline|\slshape| 	
		\\
	\textit{kursiv} 		
		\> \lstinline|\textit{kursiv}|
		\> \lstinline|\itshape| 	
		\\
	\textsc{Kapitälchen} 		
		\> \lstinline|\textsc{Kapit.}|
		\> \lstinline|\scshape| 	
		\\
	\textnormal{\parbox{7em}{Grundschrift}} 		
		\> \lstinline|\textnormal{text}|
		\> \lstinline|\normalfont| 	
		
\end{tabbing}
Hinweis: Je nach gewählter Schriftart stehen nicht alle Schriftstile zur Verfügung.

{\small
\begin{tabbing}

fcolorbox\hspace{2ex}	\= \kill% Musterzeile
	
	\fbox{fbox}	\> \lstinline|\fbox{Umrahmung}|
	\\
	\underline{underline} 	\> 
	\lstinline|\underline{}| 
	(ABER: Kein Um- \\ \> bruch möglich!) 
	\\
	
	\textcolor{red}{textcolor} \> \lstinline|\textcolor{red}{Farbiger Text}| 
	\\ \> (vgl.\, Kap.\,\ref{cha:Farben})
	\\
	
	\colorbox{lime}{colorbox} \> \lstinline|\colorbox{lime}{TEXT}|
	\\
		
	\fcolorbox{red}{lightgray}{fcolorbox} \> \lstinline|\fcolorbox{red}{lightgray}{TEXT}|
	
	\\
	hoch\textsuperscript{gestellt}	\> \lstinline|\textsuperscript{TEXT}|
	\\
	tief\textsubscript{gestellt}		\> \lstinline|\textsubscript{TEXT}|
	
\end{tabbing}
}
\noindent(siehe auch Kap.\,\ref{cha:Farben}.\nameref{cha:Farben})

%\vspace{-1\baselineskip}
\setlength{\columnseprule}{0pt}
\begin{multicols}{2}

\vspace{-2\baselineskip}
\subsubsection*{Paket \Paket{ulem}}

\lstinline|\usepackage[normalem]{ulem}|

\begin{tabbing}
	dashulin	\=  \kill % Musterzeile
\uline{uline}					\> \lstinline|\uline{}| \\
\uuline{uuline}				\> \lstinline|\uuline{}| \\
\uwave{uwave}					\> \lstinline|\uwave{}| \\
\sout{sout}						\> \lstinline|\sout{}| \\
\xout{xout}						\> \lstinline|\xout{}| \\
\dashuline{dashuline}	\> \lstinline|\dashuline{}| \\
\dotuline{dotuline}		\> \lstinline|\dotuline{}| \\
\end{tabbing}

\vspace{-2\baselineskip}
\columnbreak
\subsubsection*{Paket \Paket{fancybox}}

\begin{tabbing}
	\hspace{1ex}	\=  \kill % Musterzeile
	\> \lstinline|\usepackage{fancybox}|    \\
	\> \shadowbox{\lstinline|\shadowbox{}|} \\
	\> \doublebox{\lstinline|\doublebox{}|} \\
	\> \ovalbox{\lstinline|\ovalbox{}|} 		\\
	\> \Ovalbox{\lstinline|\Ovalbox{}|} 		\\
\end{tabbing}

\end{multicols}\setlength{\columnseprule}{0.5pt}

\negAbstand



%%-------------------------
\columnbreak
%\vspace{0.5\baselineskip} \hrule 
\subsectionstage[3]{Abstände} \label{sec:abstaende}
(siehe auch Abschn.\,\ref{cha:Laengen}.\nameref{cha:Laengen})
%
\begin{tabbing}
$x\qquad x$	x	\= \lstinline|x\qquad x|x	\= \kill
%
||				\> 										\> {\small \emph{(kein Abstand)}} 	\\
|\!|			\> \verb|\!|					\> {\small \emph{(negativ)}} \\
|\,|			\> \verb|\,|					\>  \\
|\ |			\> \lstinline|\ |			\> 	\\
|\quad |	\> \lstinline|\quad|	\> $(\hat = 1em)$	\\
|\qquad |	\> \lstinline|\qquad|	\> $(\hat = 2em)$
\end{tabbing}
%
%\begin{multicols}{2}
\subsubsection*{Läng\-en\-ein\-hei\-ten}
\vspace{-0.7\baselineskip}
\begin{tabbing}
\lstinline|1pt| |\hspace{1pt}|\\
\lstinline|1mm| |\hspace{1mm}|\\
\lstinline|1ex| |\hspace{1ex}|\\
\lstinline|1em| |\hspace{1em}|\\
\lstinline|1cm| |\hspace{1cm}|\\
\lstinline|1in| |\hspace{1in}|
\end{tabbing}

\subsubsection*{Läng\-en\-va\-ri\-ab\-len}
\vspace{-0.7\baselineskip}
\begin{lstlisting}
\baselineskip	% Zeilenabstand
\linewidth 		% Zeilenbreite
\textwidth		
\end{lstlisting}
(s.a. Abschn. \ref{sec:layout_var})

\negAbstand
%
\subsubsection*{Horizontale Ab\-stän\-de}
\vspace{-0.7\baselineskip}
\begin{lstlisting}
\,			% kleines Leerzeichen
\enspace		% Ziffernbreite
\quad				% Buchstabenhöhe
\qquad			% 2* \quad
\hfill			% variabler Abst.
\hspace{L}	% L breit
\hspace*{L}	% bleibt am Zeilen-Ende oder -Anfang Zeile erhalten.
\!	% negativ
\hspace{\widthof{my text}} % Mit Paket calc
\end{lstlisting}

\negAbstand
%
\subsubsection*{Vertikale Abstände}
\vspace{-0.7\baselineskip}
\begin{lstlisting}		
\smallskip	% etwa 1/4 Zeile
\medskip		% etwa 1/2 Zeile
\bigskip		% etwa 1 Zeile
\vfill			% var. Abst.
\vspace{L}	% L hoch
\vspace*{L}	% 
\\[1em]	% Zeilenumbruch+vert. Abst.
\end{lstlisting}
%\end{multicols}

%%-------------------------
\hrule \vspace{0.5\baselineskip}
\subsectionstage[2]{Ausrichtung}
%\vspace{-1\baselineskip}
%
\todo{Blocksatz, Flattersatz, Rauhsatz: siehe \url{uweziegenhagen.de/?p=1706}}
%
\begin{lstlisting}
\usepackage{ragged2e}
\RaggedRight	% linksbündig  (ab jetzt)
\Centering		% zentriert     (ab jetzt)
\RaggedLeft		% rechtsbündig (ab jetzt)
\begin{FlushLeft} 	... \end{FlushLeft}		% linksbündig
\begin{Center} 			... \end{Center}			% zentriert
\begin{FlushRight}	... \end{FlushRight}	% rechtsbündig

Ohne ragged2e: \raggedright \centering und \raggedleft, und flushleft, center und flushright. Eine Silbentrennung findet jedoch quasi nicht statt!
\end{lstlisting}
\negAbstand

%%% ------------------
%%%%%%%%%%%%%%%%%%%%%%%%%%%%
\subsectionstage[2]{Zeilen- / Seitenumbrüche}
%
%\vspace{-1\baselineskip}
\negAbstand
\begin{lstlisting}
\begin{samepage} 		... \end{samepage}		% kein Seitenumbruch hier ("Wunsch")
\parbox{...}{\linewidth}		% kein Seitenumbruch hier (stärker)
\begin{minipage}[position][hoehe][innen-position]{\linewidth} ... \end{minipage} % kein Seitenumbruch (wie parbox, allerdings mehr Funktionen innerhalb erlaubt.)

\mbox{ ... }				% Wie ein Zeichen behandeln. Alternative: ~ (geschütztes Leerzeichen) oder "~ (geschützter Bindestrich)
\end{lstlisting}

\subsubsection*{Geschütze Zeichen}
\begin{tabbing}
\lstinline|\,| \hspace{0.5em} \= kleines geschütztes Leerzeichen \\
\lstinline|~|		\> geschütztes ganzes Leerzeichen \\
\lstinline|"~|	\> geschützter Bindestrich 
\end{tabbing}


%%% ------------------
\columnbreak
\subsectionstage[2]{Worttrennung} \label{sec:trennung}
\subsubsection{lokal}
%
\begin{tabbing}
	\lstinline|\-| \= MMMM \= \kill
\lstinline|\-|	\hspace{1ex} \= Worttrennung \emph{nur hier} erlauben \\
\lstinline|"-|	\> Worttrennung \emph{zusätzlich} hier erl. * \\
\lstinline|""|	\> Worttrennung ohne Trennstrich erl. * \\
\lstinline|\mbox{...}| \> \> Trennung komplett verhindern \\ 
 ("`..."' wie ein eigenes Zeichen behandeln).
\end{tabbing}
{\footnotesize\slshape * Nur mit dem Paket \Paket{babel} und der Paket-Option \lstinline|[ngerman]| möglich.}
%
\subsubsection{global}
Worttrennung verbieten bzw. erlauben:\\
\lstinline|\hyphenation{Tas-se Schu-le} % Trennungsliste|

\hrule \vspace{0.5\baselineskip}
%----- EOF -----
