%%%%%%%%%%%%%%%%%%%
\setlength{\columnseprule}{0.5pt}
%\mycolumnbreak
\pagebreak

\sectionstage[2]{Mathematik}\label{cha:mathe}
\todo{Wo werden welche Pakete benötigt?}

\subsectionstage[3]{Umgebung für Mathe im Fließtext}
\begin{tabular}{c|c}
\lstinline|\begin{math} a=3 \end{math}| &  \begin{math} a=3 \end{math} \\

\lstinline|$ a=3 $| 			&  $ a=3 $ \\
\end{tabular}


%\subsectionstage[2]{Häufige Zeichen und Operatoren}


\setlength{\columnseprule}{0.5pt}
\subsection{Wichtige Operatoren}
%\begin{multicols}{2}
\begin{tabbing}
$\lim_{x\to 0}$ \hspace{2em}	\= 	\kill

$\frac{2}{3}$			\>	\lstinline|\frac{2}{3}|	\\
$\sqrt{2}$				\> 	\lstinline|\sqrt{2}|		\\
$\sqrt[3]{2}$			\> 	\lstinline|\sqrt[3]{2}|	\\
$\int \limits_{0}^{1}$			\> 	\lstinline|\int\limits_{0}^{1}|	\\
$\int\nolimits_{0}^{1}$		\> 	\lstinline|\int\nolimits_{0}^{1}|	\\
$\int_{0}^{1}$		\> 	\lstinline|\int_{0}^{1}|	\\

$\int\!\!\!\int_{D} \mathrm{d}x\,\mathrm{d}y$		
	\> 	\lstinline|\int\!\!\!\int_{D}| 	\\
	\> \lstinline|\mathrm{d}x \,| \\
	\> \lstinline|\mathrm{d}y|	\\
	
$\lim_{x\to 0}$		\> 	\lstinline|\lim_{x\to 0}|	\\
$x' x''$						\> \lstinline|x' x''| \\
$x^{a+b}$		\> 	\lstinline|x^{a+b}|	\\
$x_{a1}$		\> 	\lstinline|x_{a1}|	\\

$\hat x$							\> 	\lstinline|\hat x|	\\
$\tilde x$						\> 	\lstinline|\tilde x|	\\
$\widehat{xyz}$				\> 	\lstinline|\widehat{xyz}|	\\
$\widetilde {xyz}$		\> 	\lstinline|\widetilde {xyz}|	\\
$\overline x$					\> 	\lstinline|\overline x|	\\
$\underline x$				\> 	\lstinline|\underline x|	\\
$\vec x$							\>  \lstinline|\vec x| \\
$\overbrace{1+2}^{x}$		\> 	\lstinline|\overbrace{1+2}^{x}|	\\
$\underbrace{1+2}_{x}$		\> 	\lstinline|\underbrace{1+2}_{x}|	\\
%$n\choose k$		\> 	\lstinline|{n\choose k}|	\\[1ex]
%${x\atop y}$		\> 	\lstinline|{x\atop y}| \\[1ex]
$\overset{B}{A}$		\> 	\lstinline|\overset{B}{A}|	\\[1ex]
$\underset{B}{A}$		\> 	\lstinline|\underset{B}{A}| \\[1ex]
${\binom{a}{b}}$		\> 	\lstinline|\binom{a}{b}|
\end{tabbing}

%\end{multicols}

%%%%%%%%%%
%\genfrac{l}{r}{w}{s}{zähler}{nenner}
%genfrac dient der flexiblen Erzeugung von Brüchen. Dabei steht
    %l für einen optionalen linken Begrenzer (z.B. [),
    %r für einen optionalen rechten Begrenzer (z.B. ]),
    %w für die Dicke des Bruchstriches (width). Default ist 1pt,
    %s für den Style. Default ist 1. Gültige Werte sind 0,1,2,3 für \displaystyle, \textstyle, \scriptstyle, \scriptscriptstyle
		%%%%%%%%%%
\setlength{\columnseprule}{0.5pt}


\begin{multicols}{4}
\begin{lstlisting}
\sin
\cos
\tan
\cot
\arccos
\arcsin
\arctan
\sinh
\cosh
\tanh
\coth

\exp
\ln
\lg
\log

\min
\max 

\arg
\csc
\deg 
\det 
\dim
\gcd
\hom
\inf
\ker
\lim
\liminf
\limsup
\pmod
\Pr
\sec
\sup 
\end{lstlisting}

\end{multicols}


\subsubsection*{Paket \Paket{nicefrac}}
\lstinline|\nicefrac{2}{3}|: 
	\hspace{1em} \nicefrac{2}{3} 
	\hspace{1em} (auch außerhalb Mathe-Umgebung)

% ----- Matrizen ----- ------ -----
\subsectionstage[3]{Matrizen und Vektoren}
% ---- Matrix ----
\begin{minipage}{0.3\linewidth}
$\begin{matrix} 1 & 2 \\ 3 & 4 \end{matrix}$
\end{minipage}
%
\begin{minipage}{0.5\linewidth}
\begin{lstlisting}
\begin{matrix}
	 1 & 2 \\ 3 & 4 
\end{matrix}
\end{lstlisting}
\end{minipage}
%%%%%
\negAbstand
\begin{minipage}{0.3\linewidth}
$\begin{smallmatrix} 1 & 2 \\ 3 & 4 \end{smallmatrix}$
\end{minipage}
%
\begin{minipage}{0.5\linewidth}
\begin{lstlisting}
\begin{smallmatrix}
	 1 & 2 \\ 3 & 4
\end{smallmatrix}
\end{lstlisting}
\end{minipage}
%%%%%
\negAbstand
% ----- pmatrix -----
\begin{minipage}{0.3\linewidth}
$\begin{pmatrix} 1 & 2 \\ 3 & 4 \end{pmatrix}$
\end{minipage}
%
\begin{minipage}{0.5\linewidth}
\begin{lstlisting}
\begin{pmatrix}
	 1 & 2 \\ 3 & 4 
\end{pmatrix}
\end{lstlisting}
\end{minipage}
%%%%%
%
\verb|vmatrix:| $\begin{vmatrix} 1 \end{vmatrix}$
\qquad
\verb|Vmatrix:| $\begin{Vmatrix} 1 \end{Vmatrix}$
\qquad
\verb|bmatrix:| $\begin{bmatrix} 1 \end{bmatrix}$
\qquad
\verb|Bmatrix:| $\begin{Bmatrix} 1 \end{Bmatrix}$
%
% ----- cases -----
\begin{minipage}{0.3\linewidth}
$\begin{cases}
	%a & \text{für} & x \geq 3 \\
	%a & \text{für} & x < 3
	1 & 2 \\ 3 & 4
\end{cases}$
\end{minipage}
%
\begin{minipage}{0.5\linewidth}
\begin{lstlisting}
\begin{cases}
	 1 & 2 \\ 3 & 4 
\end{cases}
\end{lstlisting}
\end{minipage}
%
% ----- Array -----
\negAbstand
\begin{minipage}{0.3\linewidth}
$\begin{array}{cc} 1 & 2 \\ 3 & 4 \end{array}$ 
\end{minipage}
\begin{minipage}{0.5\linewidth}
\begin{lstlisting}
\begin{array}{cc}
		1 & 2 \\ 3 & 4
\end{array}
\end{lstlisting}
\end{minipage}
%%%%%


%%%%%%%%%%%%
\subsectionstage[3]{Formatierungen}
\setlength{\columnseprule}{0.5pt}
\begin{tabbing}
	$\mathbf{ABCabc}$		\= \lstinline|\mathnormal{ABCabc}| \=		\kill
	%
	$\mathnormal{ABCabc}$	\> \lstinline|\mathnormal{ABCabc}|	 \small default
	\\
	$\mathrm{ABCabc}$		\> \lstinline|\mathrm{ABCabc}|	\> \small roman
	\\
	$\mathbf{ABCabc}$		\> \lstinline|\mathbf{ABCabc}|	\> \small bold roman
	\\
	$\mathsf{ABCabc}$		\> \lstinline|\mathsf{ABCabc}|	\> \small sans serif
	\\
	$\mathit{ABCabc}$		\> \lstinline|\mathit{ABCabc}|	\> \small text italic
	\\
	$\mathtt{ABCabc}$		\> \lstinline|\mathtt{ABCabc}|	\> \small typewriter
	\\
	$\mathcal{ABCabc}$	\> \lstinline|\mathcal{ABCabc}|	\> \small calligraphic
	\\
	$\mathbb{RQZNPC}$	\> \lstinline|\mathbb{RQZNP}| 
		{\footnotesize\slshape(\Paket{amssymb})}
	\\[0.5\baselineskip]
	%

\textbf{Schriftgrößen:} 
	\\
	$\displaystyle Text$ \> \lstinline|\displaystyle Text|
		\\
	$\textstyle Text$ \> \lstinline|\textstyle Text|
		\\
	$\scriptstyle Text$ \> \lstinline|\scriptstyle Text|
		\\
	$\scriptscriptstyle Text$ \> \lstinline|\scriptscriptstyle Text|
\end{tabbing}


\subsectionstage[3]{Klammern}
\begin{tabbing}
\hspace{2.5em} \=  \hspace{3em} 
\= \hspace{2.5em} \= \hspace{3em}
\= \hspace{2.5em} \= \hspace{3em} \kill
$( \dots)$								\> \lstinline|()|	\>
$[ \dots]$								\> \lstinline|[]|	\>
$\{ \dots\}$							\> \lstinline|\{ \}	|	\\
$\lbrace \dots\rbrace$		\> \lstinline|\lbrace \rbrace|	\\
$\langle \dots\rangle$		\> \lstinline|\langle \rangle|	
\end{tabbing}
%\negAbstand
%
\begin{tabbing}
$(\frac{a}{b})$ \hspace{1ex}  \= \lstinline|(\frac{a}{b})|  \\

$\left(\frac{a}{b}\right)$  
	\> \lstinline|\left(\frac{a}{b}\right) % autom. Größe|  \\
	
$\bigl(\frac{a}{b}\bigr)$  
	\> \lstinline|\bigl(\frac{a}{b}\bigr)|  \\
	
$\Bigl(\frac{a}{b}\Bigr)$  
	\> \lstinline|\Bigl(\frac{a}{b}\Bigr)|  \\
	
$\biggl(\frac{a}{b}\biggr)$  
	\>	\begin{minipage}{0.6\linewidth}
				\lstinline|\biggl(\frac{a}{b}| 
				\lstinline|\biggr)|
			\end{minipage} \\
	
$\Biggl(\frac{a}{b}\Biggr)$  
	\> \begin{minipage}{0.6\linewidth}
				\lstinline|\Biggl(\frac{a}{b}|
				\lstinline|\Biggr)|  
			\end{minipage} \\
	
\end{tabbing}
%\ifaklein{}{
	%\end{multicols}
%}

\negAbstand
\subsectionstage[2]{Umgebungen für Mathe in separater Zeile}

\addtolength{\leftmargini}{-15pt}
\begin{itemize}
	\item {\bfseries equation oder \lstinline|\[ \]|}: Einzelne abgesetzte Gleichung
	\item multline: (Einzelne) Gleichung über mehrere Zeilen
	\item gather: 	Mehrere Gleichungen
	\item {\bfseries align}: 		Mehrere Gleichungen mit Ausrichtungspunkten
	\item flalign: 	wie align, jedoch rechts-/links-bündig
	\item *: jeweils ohne Gleichungs-Nummerierung (\mbox{\bfseries equation*}, 
	multline*, 
	gather*, 
	{\bfseries align*}, 
	flalgin*)
\end{itemize}
\addtolength{\leftmargini}{15pt}

\setlength{\columnseprule}{0.5pt } % Trennlinie für multicols


\newcommand{\myhruleB}{
	\myhrule\negAbstand
	}

%%%%% \[ \] %%%%%
\begin{multicols}{2}
\begin{lstlisting}
\[
	ab = c
\]
\end{lstlisting}
\columnbreak \hspace{0.5\baselineskip}
\[
	ab = c
\]
\end{multicols}

\myhruleB

%%%%% equation %%%%%
\begin{multicols}{2}
\begin{lstlisting}
\begin{equation} 
	ab = c \\
	d	 = ef
\end{equation}
\end{lstlisting}
\columnbreak
\begin{equation} 
	ab = c 
\end{equation}
\end{multicols}


\myhruleB

%%%%% multline %%%%%
\begin{multicols}{2}
\begin{lstlisting}
\begin{multline} 
	ab = c \\
	d	 = ef
\end{multline}
\end{lstlisting}
\begin{multline} 
	ab = c \\
	+ d \cdot ef
\end{multline}
\end{multicols}

\myhruleB

%%%%% gather %%%%%
\begin{multicols}{2}
\begin{lstlisting}
\begin{gather} 
	ab = c \\
	d	 = ef
\end{gather}
\end{lstlisting}
\begin{gather} 
	ab = c \\
	d	 = ef
\end{gather}
\end{multicols}

\myhruleB

%%%%% align %%%%%
\begin{multicols}{2}
\begin{lstlisting}
\begin{align} 
	ab&=c 	& d	&= ef \\
	g &=hi 	& j &= k 
\end{align}
\end{lstlisting}
\begin{align} 
	ab&=c 	& d	&= ef \\
	g &=hi 	& j &= k 
\end{align}
\end{multicols}

\myhruleB

%%%%% flalign %%%%%
\begin{multicols}{2}
\begin{lstlisting}
\begin{flalign} 
	ab&=c 	& d	&= ef \\
	g &=hi 	& j &= k 
\end{flalign}
\end{lstlisting}
\columnbreak
\begin{flalign} 
	ab&=c 	& d	&= ef \\
	g &=hi 	& j &= k 
\end{flalign}
\end{multicols}

\subsubsection*{Gleichungsnummer verändern}
\negAbstand\negAbstand
\begin{multicols}{2}
\begin{lstlisting}
\begin{align} 
	ab&=c  \tag{$*$} 	\\
	ab&=c  \notag 		\\
	ab&=c
\end{align}
\end{lstlisting}
\begin{align} 
	ab&=c  \tag{$*$} 	\\
	ab&=c  \notag 		\\
	ab&=c 
\end{align}
\end{multicols}

\begin{equation}
%\numbereq
	a = b
	%\label{eq:}
\end{equation}


\noindent \begin{minipage}{\linewidth}
\subsectionstage[1]{Erweiterte Formatierungen}
\negAbstand
\begin{lstlisting}
\usepackage[OPTIONEN]{amsmath}		% Erweiterter Mathematiksatz
	% Option leqno:  Gleichungsnummer links
	% Option requno: Gleichungsnummer rechts
	% Option fleqn:  Formel mit Einzug. Einzug in Präambel definierbar mit: \setlength\mathindent{5mm}
	
% Abstände Matheumgebungen
\setlength\abovedisplayshortskip{0pt}		% ?
\setlength\belowdisplayshortskip{10pt}	% ?
\setlength\abovedisplayskip{0pt}				% kein Abstand vor Gleichungsumgebungen
\setlength\belowdisplayskip{20pt}				% ?
\end{lstlisting}

\subsubsection*{Mathe in Überschriften \stage{1}}
Schriftart anpassen (nur KOMA-Klassen):
\begin{lstlisting}
	\renewcommand*{\sectfont}{\sffamily\bfseries\boldmath}
\end{lstlisting}

\end{minipage}


\hrule \vspace{0.5\baselineskip}
%----- EOF -----


