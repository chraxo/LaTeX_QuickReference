%%%%%%%%%%%%%%%%%%%%%%%
\pagebreak

\setlength{\columnseprule}{0.5pt}
%\addsec{\partcolor---Farben---}
\sectionstage[3]{Farben \Paket{(color}, \Paket{xcolor)}}
\label{cha:Farben}

\subsectionstage[3]{Farben-Befehle}
\negAbstand
\begin{lstlisting}
\pagecolor{FARBE} 	% Seitenfarbe ändern
\textcolor{FARBE}{TEXT}	% Farbiger Text
\colorbox{FARBE}{TEXT}	% Farbig hinterlegt
\fcolorbox{RAHMENFARBE}{HINTERGRUNDFARBE}{TEXT}
\color{FARBE}			% Textfarbe ab jetzt
\end{lstlisting}

%\newcommand{\cb}[1]{\colorbox{#1}{\textcolor{#1}{T}} \mbox{#1} \linebreak}
\newcommand{\cb}[1]{\textcolor{#1}{\rule{1.1em}{2.2ex}} \mbox{\footnotesize #1} \newline}

\setlength{\columnseprule}{0pt}
\noindent\begin{minipage}{\linewidth}

\subsectionstage[4]{Vordefinierte Standard-Farben\hspace{-0.8ex}}
\vspace{-0.5\baselineskip}
\begin{lstlisting}
\usepackage{xcolor}
\end{lstlisting} \vspace{-1\baselineskip}


\begin{multicols}{3}
\noindent
\cb{yellow}
\cb{orange}
\cb{red}
\cb{purple}
\cb{pink}
%
\cb{magenta}
\cb{violet}
\cb{blue}
\cb{cyan}
\cb{teal}
%
\cb{green}
\cb{lime}
\cb{olive}
%
\cb{brown}
\cb{black} \cb{darkgray} \cb{gray} \cb{lightgray} \cb{white}
\end{multicols}
%\end{minipage}
\end{minipage}

\posAbstand

\noindent\begin{minipage}{\linewidth}
\subsectionstage[4]{Weitere vordefinierte Farben}

\lstinline|\usepackage[dvipsnames]{xcolor}|

\negAbstand

\setlength{\columnseprule}{0pt}
	\begin{multicols}{2}
\noindent
%\newcommand{\cb}[1]{\colorbox{#1}{\textcolor{#1}{T}} \mbox{#1} \linebreak} % nicht klar, warum hier erneute Definition notwendig ist. Anderfalls gibt es jedenfalls einen Error.
\cb{Yellow}
\cb{Goldenrod}
\cb{Dandelion}
\cb{YellowOrange}
\cb{BurntOrange}
\cb{Orange}
\cb{Peach}
\cb{Melon}
\cb{Apricot}
\cb{Tan}
\cb{RedOrange}
\cb{Red}
\cb{OrangeRed}
\cb{Maroon}
\cb{Mahogany}
\cb{RawSienna}
\cb{Brown}
\cb{Sepia}
\cb{RedViolet}
%
\cb{RubineRed}
\cb{Rhodamine}
\cb{CarnationPink}
\cb{Salmon}
\cb{WildStrawberry}
\cb{Magenta}
\cb{VioletRed}
\cb{Lavender}
\cb{Thistle}
\cb{Orchid}
\cb{Purple}
\cb{Fuchsia}
\cb{DarkOrchid}
\cb{Mulberry}
\cb{Plum}

\end{multicols}
\end{minipage}

\begin{minipage}{\linewidth}
\begin{multicols}{2}

\cb{Periwinkle}
\cb{Violet}
\cb{RoyalPurple}
\cb{CadetBlue}
\cb{BlueViolet}
\cb{Blue}
\cb{NavyBlue}
\cb{RoyalBlue}
\cb{Cerulean}
\cb{Cyan}
\cb{ProcessBlue}
\cb{CornflowerBlue}
\cb{SkyBlue}
\cb{MidnightBlue}
\cb{Turquoise}
\cb{Aquamarine}
\cb{BlueGreen}
\cb{TealBlue}
\cb{Emerald}
\cb{JungleGreen}
\cb{PineGreen}
\cb{SeaGreen}
\cb{Green}
\cb{ForestGreen}
\cb{OliveGreen}
\cb{GreenYellow}
\cb{SpringGreen}
\cb{YellowGreen}
\cb{LimeGreen}
\cb{Black}
\cb{Gray}
\cb{White}
\end{multicols}
\end{minipage}
\setlength{\columnseprule}{0.5pt}

\subsectionstage[3]{Benutzerdefinierte Farben}
%\negAbstand
Erlaubte Farmodelle: natural, rgb, cmy, cmyk, hsb, gray, RGB, HTML, HSB, Gray.
\begin{lstlisting}
\textcolor[rgb]{0.1,0.2,0.3}{TEXT}

\definecolor{MYCOL}{rbg} {0.5,0.4,0.7} % red, green, blue [0,1]
\definecolor{MYCOL}{RGB} {60,23,144}		% Red, Green, Blue [0,255]
%\definecolor{MYCOL}{cmy} {0.5,0.4,0.7}	% cyan, magenta, yellow [0,1]
\definecolor{MYCOL}{cmyk} {0.5,0.4,0.7,0.3}	% cyan, magenta, yellow, black [0,1]
\definecolor{MYCOL}{gray} {0.4} % [0,1]
\definecolor{MYCOL}{HTML} {AFFE90}          %[000000,..., FFFFFF]
\definecolor{MYCOL}{hsb} {0.5,0.4,0.7} % hue,saturation,brightness [0,1]
\end{lstlisting}
Empfehlung: HSB-Farbraum (auch "`HSV"'). {\footnotesize (siehe \url{https://de.wikipedia.org/wiki/HSV-Farbraum})}

%\definecolor{javared}{rgb}{0.6,0,0} % for strings
%\definecolor{javagreen}{rgb}{0.25,0.5,0.35} % comments
%\definecolor{javapurple}{rgb}{0.5,0,0.35} % keywords
%\definecolor{javadocblue}{rgb}{0.25,0.35,0.75} % javadoc

%-------------------
\negAbstand
\subsectionstage[2]{Farbige Mathematik}
In Mathe-Umgebung:
\begin{lstlisting}
\textcolor{red]{...}	% wie gewohnt
\color{red}						% wie gewohnt
Inline auch: \textcolor{red}{$Mathe$}
oder:
\textcolor{red}{
	\parbox{\linewidth}{
		\begin{equation} ...
		\end{equation}
	}
}
\end{lstlisting}
\hrule \vspace{0.5\baselineskip}
%----- EOF -----