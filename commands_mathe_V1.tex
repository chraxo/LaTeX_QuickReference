%----- d/dt  und  d{}/dt -----
\newcommand{\ddt}[1][]	% z.B. \ddt[\phi] oder \ddt
	{\frac{ \mathrm{d}#1 }{\mathrm{d}t}}

%----- dt -----
\newcommand{\dt}
	{\mathrm{d}t}
	
%----- d{} -----
\newcommand{\diff}
	[1]
	{\mathrm{d}#1}

%----- dphi -----	
\newcommand{\dphi}
	{\mathrm{d}\phi}
	

%----- Mathe-Abk�rzungen -----
\newcommand{\Ln}{\mathrm{Ln}}
\newcommand{\arcosh}{\mathrm{arcosh}\,}


%----- Mengen-Symbole -----
\newcommand{\N}{\mathbb{N}}	% Nat�rliche Zahlen (0,1,2,3,..)
\newcommand{\Z}{\mathbb{Z}} % Ganze Zahlen (..,-2,-1,0,1,2,..)
\newcommand{\Q}{\mathbb{Q}} % Rationale Zahlen (Quotienten aus Z)
\newcommand{\R}{\mathbb{R}} % Reelle Zahlen (Q + sqrt(2) etc.)
\newcommand{\C}{\mathbb{C}} % Komplexe Zahlen (1+3*i)
%% Info: \N \in \Z \in \Q \in \R \in \C