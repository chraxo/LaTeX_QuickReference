% befehle.tex
% Eigene Befehle

\newcommand{\Abb}	{Abb.\,}
\newcommand{\etal}	{et\,al.\xspace}
\newcommand{\zB}	{z.\,B.\xspace}
\newcommand{\vgl}	{vgl.\,}
\newcommand{\pfeil}	{$\rightarrow$\xspace}
\newcommand{\Pfeil}	{$\Rightarrow$\xspace}
\newcommand{\Gl}	{Gl.\,}					% Gleichung
\newcommand{\tabbingrule}	{\rule{\linewidth}{0.5\arrayrulewidth}}
%\newcommand{\}	{}
%\newcommand{\}	{}
%\newcommand{\}	{}
%\newcommand{\}	{}
%\newcommand{\}	{}



%----- Einheiten -----
\newcommand{\cl}{\,cl\xspace}
\newcommand{\g}{\,g\xspace}
\newcommand{\mg}{\,mg\xspace}
\newcommand{\ml}{\,ml\xspace}
\newcommand{\Lit}{\,L\xspace}
\newcommand{\kJ}{\,kJ\xspace}
\newcommand{\kcal}{\textcolor{red}{\,kcal}}

\newcommand{\EL}{\,EL\xspace}
\newcommand{\TL}{\,TL\xspace}

%----- Blindtext -----
\newcommand{\Bla}{
	Dies hier ist ein Blindtext zum Testen von Textausgaben. Wer diesen Text liest, ist selbst schuld.
	Der Text gibt lediglich den Grauwert der Schrift an.
	Ist das wirklich so?
	Ist es gleichgültig, ob ich schreibe: „Dies ist ein Blindtext“ oder „Huardest gefburn“? Kjift mitnichten!
	Ein Blindtext bietet mir wichtige Informationen.
	}
\newcommand{\bla}{
	Dies hier ist ein Blindtext zum Testen von Textausgaben.
	}


% ---- red ----
\newcommand
	{\red}
	{\color{red}}
\newcommand
	{\txtred}[1]
	{\textcolor{red}{#1}}
\newcommand
	{\bgred}[1]
	{\colorbox{Salmon}{#1}}

% ---- orange ----
\newcommand
	{\orange}
	{\color{orange}}
\newcommand
	{\txtorange}[1]
	{\textcolor{orange}{#1}}
\newcommand
	{\bgorange}[1]
	{\colorbox{orange}{#1}}

% ---- yellow ----
\newcommand
	{\yellow}
	{\color{yellow}}
\newcommand
	{\txtyellow}[1]
	{\textcolor{yellow}{#1}}
\newcommand
	{\bgyellow}[1]
	{\colorbox{yellow}{#1}}

	
% ----green  ----
\newcommand
	{\green}
	{\color{Green}}
\newcommand
	{\txtgreen}[1]
	{\textcolor{Green}{#1}}
\newcommand
	{\bggreen}[1]
	{\colorbox{lime}{#1}}
	
% ---- blue ----
	\newcommand
	{\blue}
	{\color{Blue}}
\newcommand
	{\txtblue}[1]
	{\textcolor{Blue}{#1}}
\newcommand
	{\bgblue}[1]
	{\colorbox{Cerulan}{#1}}
	
% ---- gray ----
\newcommand
	{\gray}
	{\color{gray}}
\newcommand
	{\txtgray}[1]
	{\textcolor{gray}{#1}}
\newcommand
	{\bggray}[1]
	{\colorbox{lightgray}{#1}}
	
% ----  ----

%----- d/dt  und  d{}/dt -----
\newcommand{\ddt}[1][]	% z.B. \ddt[\phi] oder \ddt
	{\frac{ \mathrm{d}#1 }{\mathrm{d}t}}

%----- dt -----
\newcommand{\dt}
	{\mathrm{d}t}
	
%----- d{} -----
\newcommand{\diff}
	[1]
	{\mathrm{d}#1}

%----- dphi -----	
\newcommand{\dphi}
	{\mathrm{d}\phi}
	

%----- Mathe-Abk�rzungen -----
\newcommand{\Ln}{\mathrm{Ln}}
\newcommand{\arcosh}{\mathrm{arcosh}\,}


%----- Mengen-Symbole -----
\newcommand{\N}{\mathbb{N}}	% Nat�rliche Zahlen (0,1,2,3,..)
\newcommand{\Z}{\mathbb{Z}} % Ganze Zahlen (..,-2,-1,0,1,2,..)
\newcommand{\Q}{\mathbb{Q}} % Rationale Zahlen (Quotienten aus Z)
\newcommand{\R}{\mathbb{R}} % Reelle Zahlen (Q + sqrt(2) etc.)
\newcommand{\C}{\mathbb{C}} % Komplexe Zahlen (1+3*i)
%% Info: \N \in \Z \in \Q \in \R \in \C

% Formatierungsbefehle
\newcommand{\Paket}[1]{\textcolor{purple}{#1}}
\newcommand{\Befehl}[1]{\textcolor{NavyBlue}{#1}}	% (blue, NavyBlue)
\newcommand{\Umgebung}[1]{\textcolor{teal}{#1}}
\newcommand{\Laenge}[1]{\textcolor{teal}{#1}}		
\newcommand{\notiz}{\color{Cerulean}}

% Vereinfachungs-Befehle
\newcommand{\mytitleformat}[1]{\texorpdfstring{#1}{}}
	% verhindert Fehler, wenn Formatierungen im Titel verwendet werden: \section{ \titelformat{\color{red}} TEXT }
\newcommand{\scite}[2]{\cite[S.\,#2]{#1}}	% \scite{Jahne12}{123\,f}

\newcommand{\mycolumnbreak}{\columnbreak} % (\columnbreak,\newpage)
\newcommand{\myHspace}{\hspace{1ex}} % Abstand in Listen Mathematische Symbole
\newcommand{\myhrule}
	{	\vspace{-.5\baselineskip} \hrule }
	
\newcommand{\mminipage}[2]{\begin{minipage}{#1}#2\end{minipage}}
\newcommand{\befehlsformat}[1]
	{%
	{\ttfamily\small\color{NavyBlue} \textbackslash{#1}}%
	}
\newcommand{\todo}[1]{{\textcolor{purple}{TODO: \textasteriskcentered} \color{gray}~#1~\textcolor{purple}{\textasteriskcentered}}}
	
%----- Abstände -----
\newcommand{\negAbstand}{\vspace{-0.55\baselineskip}} % negativer Abstand. Zweck: Abstand zwischen Überschrift und lstlisting-Umgebung oder ähnlichem verringern.
\newcommand{\posAbstand}{\vspace{0.55\baselineskip}}
\newcommand{\negspace}{\negAbstand} % Synonym --> Coolere Befehl
\newcommand{\posspace}{\posAbstand} % Synonym --> Coolere Befehl
\newcommand{\notizenplatz}{\vspace{5\baselineskip}}
\newcommand{\notizenplatzx}{\vspace*{5\baselineskip}}

%-----  ------
%%Befehl \mycirc{..} für Kreis um Buchstaben definieren (usepackage{tikz} notwendig)
%\newcommand*{\mycirc}[1]{ 
	%\begin{tikzpicture}[baseline=(C.base)]%
    %\!\!\!% negativer Abstand
		%\node[draw,circle,inner sep=1pt](C) {#1};
		%\!\!\!%negativer Abstand
  %\end{tikzpicture}
%}


% Development stages (wikibooks.org nachempfunden. vgl. [en.wikibooks.org/wiki/Help:Development_stages])
\newlength{\stageHeight}
\setlength{\stageHeight}{1.5ex}
%\settoheight{\stageHeight}{d}
\newcommand{\stage}[1]{
	\ifcase #1 % 0
		\texorpdfstring{%
			\settoheight{\stageHeight}{R}%
			\includegraphics[height=\stageHeight,%
			trim={0 0.3mm 0 0}]{Bilder/stage0}%
		}{}%
	\or % 1
		\texorpdfstring{%
			\settoheight{\stageHeight}{R}%
			\includegraphics[height=\stageHeight,%
			trim={0 0.3mm 0 0}]{Bilder/stage1}%
		}{}%
	\or % 2
		\texorpdfstring{%
			\settoheight{\stageHeight}{R}%
			\includegraphics[height=\stageHeight,
			trim={0 0.3mm 0 0}]{Bilder/stage2}%
		}{}%
	\or % 3
		\texorpdfstring{%
			\settoheight{\stageHeight}{R}%
			\includegraphics[height=\stageHeight,%
			trim={0 0.3mm 0 0}]{Bilder/stage3}%
		}{}%
	\or % 4
		\texorpdfstring{%
			\settoheight{\stageHeight}{R}%
			\includegraphics[height=\stageHeight,%
			trim={0 0.3mm 0 0}]{Bilder/stage4}%
		}{}%
	\else		% >4
		\textcolor{red}{ND}
	\fi
}
% Beschreibung:
%%\begin{description}
	%%\item[\stage{0} Neu]  				Kaum Text
	%%\item[\stage{1} Erstentwurf]  Keine klare Struktur
	%%\item[\stage{2} In Arbeit]		Mittl. -- großer Änd.-bed.
	%%\item[\stage{3} Fortgeschritten]	Geringer Änd.-bed.
	%%\item[\stage{4} Fertig] 	Kein Änderungsbedarf
%%\end{description}
%%Symbole entnommen aus [Wikibooks.org "`Development Stages"' (\url{en.wikibooks.org/wiki/Help:Development_stages})]
\newcommand{\sectionstage}[2][-1]{%
	\section[#2]{#2\stage{#1}}% 	\sectionstage{2}{TITEL}
}
\newcommand{\subsectionstage}[2][-1]{% 
	\subsection[#2]{#2\stage{#1}}% 			\subsectionstage{2}{TITEL}
}
\newcommand{\subsubsectionstage}[2][-1]{% 
	\subsubsection[#2]{#2\stage{#1}}% 	\subsubsectionstage{2}{TITEL}
}
%
\newcommand{\sectionstageS}[2][-1]{% X anstatt von * (irgendwie nicht möglich)
	\section*{#2\stage{#1}}% 	\subsubsectionstage{2}{TITEL}
}
\newcommand{\subsectionstageS}[2][-1]{% 
	\subsection*{#2\stage{#1}}% 	\subsubsectionstage{2}{TITEL}
}
\newcommand{\subsubsectionstageS}[2][-1]{% 
	\subsubsection*{#2\stage{#1}}% 	\subsubsectionstage{2}{TITEL}
}


%%%%%%%%%%%%%%%%%%%%%%%%%%%%%%%%%